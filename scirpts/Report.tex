% Options for packages loaded elsewhere
\PassOptionsToPackage{unicode}{hyperref}
\PassOptionsToPackage{hyphens}{url}
%
\documentclass[
]{article}
\usepackage{amsmath,amssymb}
\usepackage{iftex}
\ifPDFTeX
  \usepackage[T1]{fontenc}
  \usepackage[utf8]{inputenc}
  \usepackage{textcomp} % provide euro and other symbols
\else % if luatex or xetex
  \usepackage{unicode-math} % this also loads fontspec
  \defaultfontfeatures{Scale=MatchLowercase}
  \defaultfontfeatures[\rmfamily]{Ligatures=TeX,Scale=1}
\fi
\usepackage{lmodern}
\ifPDFTeX\else
  % xetex/luatex font selection
\fi
% Use upquote if available, for straight quotes in verbatim environments
\IfFileExists{upquote.sty}{\usepackage{upquote}}{}
\IfFileExists{microtype.sty}{% use microtype if available
  \usepackage[]{microtype}
  \UseMicrotypeSet[protrusion]{basicmath} % disable protrusion for tt fonts
}{}
\makeatletter
\@ifundefined{KOMAClassName}{% if non-KOMA class
  \IfFileExists{parskip.sty}{%
    \usepackage{parskip}
  }{% else
    \setlength{\parindent}{0pt}
    \setlength{\parskip}{6pt plus 2pt minus 1pt}}
}{% if KOMA class
  \KOMAoptions{parskip=half}}
\makeatother
\usepackage{xcolor}
\usepackage[margin=1in]{geometry}
\usepackage{color}
\usepackage{fancyvrb}
\newcommand{\VerbBar}{|}
\newcommand{\VERB}{\Verb[commandchars=\\\{\}]}
\DefineVerbatimEnvironment{Highlighting}{Verbatim}{commandchars=\\\{\}}
% Add ',fontsize=\small' for more characters per line
\usepackage{framed}
\definecolor{shadecolor}{RGB}{248,248,248}
\newenvironment{Shaded}{\begin{snugshade}}{\end{snugshade}}
\newcommand{\AlertTok}[1]{\textcolor[rgb]{0.94,0.16,0.16}{#1}}
\newcommand{\AnnotationTok}[1]{\textcolor[rgb]{0.56,0.35,0.01}{\textbf{\textit{#1}}}}
\newcommand{\AttributeTok}[1]{\textcolor[rgb]{0.13,0.29,0.53}{#1}}
\newcommand{\BaseNTok}[1]{\textcolor[rgb]{0.00,0.00,0.81}{#1}}
\newcommand{\BuiltInTok}[1]{#1}
\newcommand{\CharTok}[1]{\textcolor[rgb]{0.31,0.60,0.02}{#1}}
\newcommand{\CommentTok}[1]{\textcolor[rgb]{0.56,0.35,0.01}{\textit{#1}}}
\newcommand{\CommentVarTok}[1]{\textcolor[rgb]{0.56,0.35,0.01}{\textbf{\textit{#1}}}}
\newcommand{\ConstantTok}[1]{\textcolor[rgb]{0.56,0.35,0.01}{#1}}
\newcommand{\ControlFlowTok}[1]{\textcolor[rgb]{0.13,0.29,0.53}{\textbf{#1}}}
\newcommand{\DataTypeTok}[1]{\textcolor[rgb]{0.13,0.29,0.53}{#1}}
\newcommand{\DecValTok}[1]{\textcolor[rgb]{0.00,0.00,0.81}{#1}}
\newcommand{\DocumentationTok}[1]{\textcolor[rgb]{0.56,0.35,0.01}{\textbf{\textit{#1}}}}
\newcommand{\ErrorTok}[1]{\textcolor[rgb]{0.64,0.00,0.00}{\textbf{#1}}}
\newcommand{\ExtensionTok}[1]{#1}
\newcommand{\FloatTok}[1]{\textcolor[rgb]{0.00,0.00,0.81}{#1}}
\newcommand{\FunctionTok}[1]{\textcolor[rgb]{0.13,0.29,0.53}{\textbf{#1}}}
\newcommand{\ImportTok}[1]{#1}
\newcommand{\InformationTok}[1]{\textcolor[rgb]{0.56,0.35,0.01}{\textbf{\textit{#1}}}}
\newcommand{\KeywordTok}[1]{\textcolor[rgb]{0.13,0.29,0.53}{\textbf{#1}}}
\newcommand{\NormalTok}[1]{#1}
\newcommand{\OperatorTok}[1]{\textcolor[rgb]{0.81,0.36,0.00}{\textbf{#1}}}
\newcommand{\OtherTok}[1]{\textcolor[rgb]{0.56,0.35,0.01}{#1}}
\newcommand{\PreprocessorTok}[1]{\textcolor[rgb]{0.56,0.35,0.01}{\textit{#1}}}
\newcommand{\RegionMarkerTok}[1]{#1}
\newcommand{\SpecialCharTok}[1]{\textcolor[rgb]{0.81,0.36,0.00}{\textbf{#1}}}
\newcommand{\SpecialStringTok}[1]{\textcolor[rgb]{0.31,0.60,0.02}{#1}}
\newcommand{\StringTok}[1]{\textcolor[rgb]{0.31,0.60,0.02}{#1}}
\newcommand{\VariableTok}[1]{\textcolor[rgb]{0.00,0.00,0.00}{#1}}
\newcommand{\VerbatimStringTok}[1]{\textcolor[rgb]{0.31,0.60,0.02}{#1}}
\newcommand{\WarningTok}[1]{\textcolor[rgb]{0.56,0.35,0.01}{\textbf{\textit{#1}}}}
\usepackage{graphicx}
\makeatletter
\def\maxwidth{\ifdim\Gin@nat@width>\linewidth\linewidth\else\Gin@nat@width\fi}
\def\maxheight{\ifdim\Gin@nat@height>\textheight\textheight\else\Gin@nat@height\fi}
\makeatother
% Scale images if necessary, so that they will not overflow the page
% margins by default, and it is still possible to overwrite the defaults
% using explicit options in \includegraphics[width, height, ...]{}
\setkeys{Gin}{width=\maxwidth,height=\maxheight,keepaspectratio}
% Set default figure placement to htbp
\makeatletter
\def\fps@figure{htbp}
\makeatother
\setlength{\emergencystretch}{3em} % prevent overfull lines
\providecommand{\tightlist}{%
  \setlength{\itemsep}{0pt}\setlength{\parskip}{0pt}}
\setcounter{secnumdepth}{-\maxdimen} % remove section numbering
\ifLuaTeX
  \usepackage{selnolig}  % disable illegal ligatures
\fi
\usepackage{bookmark}
\IfFileExists{xurl.sty}{\usepackage{xurl}}{} % add URL line breaks if available
\urlstyle{same}
\hypersetup{
  pdftitle={Amazon Product Co-Purchasing Network Analysis},
  pdfauthor={Mu Niu, Kai Wa Ho, Jingtang},
  hidelinks,
  pdfcreator={LaTeX via pandoc}}

\title{Amazon Product Co-Purchasing Network Analysis}
\author{Mu Niu, Kai Wa Ho, Jingtang}
\date{2024-06-06}

\begin{document}
\maketitle

\subsection{Introduction}\label{introduction}

In this project, we conduct a comprehensive social network analysis on a
dataset derived from Amazon's product co-purchasing records. The raw
edge data, which was collected on June 1, 2003, from the Amazon website,
includes 403,394 nodes representing individual products and 3,387,388
unweighted directed edges indicating frequently co-purchased product
pairs. The direction of an edge signifies the order of purchase, with an
outward edge from product A to product B indicating that product B is
frequently purchased after product A. This dataset is available at
\href{https://snap.stanford.edu/data/amazon0601.html}{\textbf{Amazon
Product Co-Purchasing Network}}\textbf{.}

Additionally, the raw nodes data was collected by crawling Amazon's
website and encompasses metadata and review information for 548,552
different products, including books, music, DVDs, and video tapes. This
dataset provides detailed information such as the product title, sales
rank, list of similar products, detailed product categorization, and
reviews. This data was collected in the summer of 2006, and it can be
accessed at
\href{https://snap.stanford.edu/data/amazon-meta.html}{\textbf{Amazon
Product Metadata}}\textbf{.}

After a thorough data cleaning process, we integrated the edge data and
node information data, resulting in an igraph object containing 398,688
nodes. Each node has four attributes: ID (unique product ID), Title
(product name), Group (class the product belongs to), and Category
(subcategory of the class).

The primary objectives of this analysis are to understand the
relationships between products, identify co-purchasing patterns and
customer shopping behaviors through network data visualization, and
explore various network, node, and edge metrics. We aim to interpret
these metrics within the network context, apply community detection
algorithms, and analyze the resulting communities. Additionally, we will
examine the network's adjacency matrix, reorder vertices to highlight
patterns, and observe changes in the matrix to identify clearer patterns
based on community or connected components.

\subsection{Methodology}\label{methodology}

In this project, we followed a structured approach to process the data,
analyze the network, and visualize the results.The data processing
procedures began with mapping the nodes information dataset, which was a
text file with issues of missing data and inconsistent formats, into a
dataframe with four columns: ID, Title, Group, and Category.
Subsequently, we filtered out all edges in the edge dataset that
connected nodes with missing IDs, product titles, or product groups.
This ensured that only valid and complete data was included in the
analysis. The mapped nodes information data and filtered edge data was
then integrated into an igraph object containing 398,688 nodes with four
vertex attributes: ID, Title, Group, and Category. The data cleaning
process employed packages such as \texttt{dplyr} for data manipulation
and \texttt{igraph} for constructing and managing the igraph object in
R.

For sampling methodology, we generated 4 induced subgraphs by using a
node-centric approach. For each sub-network, we randomly selected one
node as the starting point and retrieved all nodes connected to it. We
then iteratively expanded our sample by including nodes connected to the
current set of nodes, continuing this process until the specified number
of nodes was reached. This sampling method was designed to capture
densely connected nodes, facilitating better visualization and
understanding of the relationships between products.

This comprehensive methodology enabled us to effectively analyze and
interpret the Amazon product co-purchasing network, uncovering valuable
insights into customer shopping behaviors and product relationships.

\subsection{Analysis}\label{analysis}

After sampling four sub-networks, each formed by using our previously
described sampling method and starting with nodes from the book, music,
video, and DVD groups respectively, we visualized the subgraphs. This
visualization was performed using the base plot and Plotly package in R
to generate both static and interactive plots.

The visualizations revealed several interesting patterns in
co-purchasing behavior. For instance, we observed that individuals who
purchase education books also tend to listen to R\&B, rap, and hip-hop
music. This correlation might suggest a demographic overlap between
students or educators and fans of these music genres. Additionally, we
found that people who buy nonfiction books also favor history books,
which is logical as both are grounded in factual content. Travel book
enthusiasts were also seen to purchase audiobooks, likely due to the
convenience of not carrying physical copies while traveling. Similarly,
customers who buy children's books often buy audiobooks, possibly
because audiobooks provide a convenient way for babysitters to engage
children. In the DVD group, those who enjoy comedy DVDs also show a
diverse taste in music, purchasing DJ music, pop, rock, and metal. An
intriguing finding was that several memoir nodes were linked to R\&B and
DJ music, which might suggest that people prefer listening to these
genres while reading memoirs. Additionally, we found that individuals
interested in business and investment books also showed an interest in
books on arts and photography.

These visualizations enhanced our understanding of the relationships
between products and allowed us to identify distinct co-purchasing
patterns. The interactive plots, containing more detailed information,
can be accessed
\href{https://amazon-product-sna-visualization.netlify.app/}{\textbf{here}}\textbf{.}
By analyzing these subgraphs, we gained valuable insights into customer
behavior and product affinities, which can be leveraged for better
marketing strategies and product recommendations.

\subsubsection{Metrics and Community
Detection}\label{metrics-and-community-detection}

\subsubsection{Condensed Summary:}\label{condensed-summary}

The study of co-purchase networks offers significant insights into
customer behavior, product relationships, and marketing strategies. This
analysis focuses on the co-purchase networks of Music, Book, Video, and
DVD categories. Each category's network structure is examined through
its adjacency matrix and various metrics to understand the connectivity
and influential nodes.

In the Music network, a density of 0.237 indicates moderate connectivity
with frequent co-purchases. Key metrics such as average path length,
degree distribution, and centrality measures (betweenness, closeness,
and eigenvector) reveal a balanced network with some influential hubs
and strong co-purchase relationships.

The Book network shows lower density (0.137), reflecting a more
dispersed structure with significant but less frequent connections. Its
metrics indicate moderate connectivity and a few key nodes maintaining
network cohesion.

For the Video network, a density of 0.189 suggests moderate connectivity
with a compact structure, characterized by average path lengths and
strong, though moderately distributed, influential nodes.

Community detection through algorithms like Walktrap, Infomap, and
FastGreedy offer deeper insights into each network's community dynamics.
The Walktrap algorithm, which uses random walks, and Infomap, focusing
on information flow compression, both identify smaller, tighter
communities. FastGreedy, through hierarchical merging, captures broader
community structures with slightly less sharp boundaries.

Each algorithm highlights how products are grouped and connected within
the network, with Walktrap and Infomap providing more detailed, smaller
communities and FastGreedy offering a broader view. The stability of
community structures across different networks and algorithms
underscores the robustness of the network's configuration, showing that
recommendations and co-purchases propagate through stable, well-defined
clusters.

Overall, these analyses not only map out the internal dynamics of
product co-purchase networks but also provide actionable insights for
optimizing product recommendations and marketing strategies.

\subsubsection{Adjacency Matrix\&Reordering
Nodes}\label{adjacency-matrixreordering-nodes}

An adjacency matrix is an essential tool for visualizing and analyzing
relationships in social networks. Due to report length constraints, I
will first examine the original and reordered matrices of the combined
network, which integrates four distinct networks. Then, focusing on the
music network as a representative.

Firstly, we began by examining the combined adjacency matrix for four
distinct product categories: Music, Book, Video, and DVD. Each cell in
the matrix represents a co-purchase relationship between products, with
grey cells indicating connections. The matrices showed dense
intra-network connections and minimal inter-network interactions,
highlighting the stability and strong internal community structures
within each category.

For a deeper analysis, we focused on the music network, where we
explored its adjacency matrix and employed community detection
algorithms such as Walktrap, Infomap, and FastGreedy. These algorithms
allowed us to reorder the network, revealing clusters centered around
influential nodes like 332118--``Movimiento Music,'' 251117--``G.I.
Blues,'' and 240889--``De La Soul Is Dead.'' The consistent appearance
of these clusters across different algorithms underscores the network's
inherent robustness and stable structure.

Notably, Walktrap and Infomap effectively identified smaller,
tightly-knit communities, reflecting specific co-purchasing behaviors.
In contrast, FastGreedy captured broader community structures but with
less distinct separations. This suggests that while the algorithms
detect similar clusters, their mechanisms define these clusters with
varying sharpness. Additionally, community detection algorithms seem to
aggregate broader communities, which may merge smaller, distinct
clusters into larger ones. This can reduce the apparent number of
clusters which we see in the original matrix.

Our findings demonstrate how product recommendations and purchases
propagate within the network, forming cohesive communities around
frequently co-purchased items. This aligns with principles from viral
marketing, where understanding these dynamics is crucial for predicting
future purchases and optimizing marketing strategies. The observed
patterns and clusters provide valuable insights into consumer behavior
and the interconnectedness of products within the network, emphasizing
the importance of community dynamics in network analysis. Given the
consistency of these findings, the detailed analysis of the other
networks can be inferred to show similar dynamics, justifying a focus on
the music network for this study.

\subsection{Conclusion}\label{conclusion}

We analyzed the Amazon product co-purchasing network to understand its
structure and dynamics. Initially, we looked at the basic
characteristics of the network, which showed moderate density with
significant co-purchasing connections among music products. This setup
allows us to see clear patterns and clusters, indicating that while many
products are connected, they're not all directly linked.

Our study used community detection algorithms like Walktrap, Infomap,
and FastGreedy to identify community structures within the network.
Walktrap and Infomap were particularly good at finding smaller,
closely-knit groups, showing specific buying behaviors among certain
products. On the other hand, FastGreedy helped us see larger community
structures, providing a broader view of how the network is organized.

We also used visualizations like reordered adjacency matrices to
illustrate the network's dense connections and the formation of
communities around frequently bought items. These visual tools confirmed
that our community detection methods were effective and that the network
structure was both robust and stable.

Our analysis of the Amazon product co-purchasing network provides
valuable insights for enhancing marketing strategies. We identified
clusters and key products that illustrate how recommendations and
purchases propagate through the network. This information is crucial for
targeting marketing efforts effectively and optimizing product
recommendations. The study underscores the importance of understanding
community dynamics within these networks, offering essential guidance
for developing targeted marketing approaches and improving
recommendation systems in e-commerce and marketing.

\subsection{References}\label{references}

\subsubsection{Dataset:}\label{dataset}

\begin{itemize}
\item
  \href{https://snap.stanford.edu/data/amazon0601.html}{\textbf{Amazon
  Product Co-Purchasing Network}}\textbf{:} The edge dataset
  representing products and their co-purchasing patterns on Amazon.
\item
  \href{https://snap.stanford.edu/data/amazon-meta.html}{\textbf{Amazon
  Product Metadata}}\textbf{:} Metadata for products listed in the
  Amazon Product Co-Purchasing Network.
\end{itemize}

\subsubsection{Academic Reference:}\label{academic-reference}

\begin{itemize}
\tightlist
\item
  J. Leskovec, L. Adamic, and B. Adamic.
  \href{https://doi.org/10.1145/1232722.1232727}{The Dynamics of Viral
  Marketing}. ACM Transactions on the Web (ACM TWEB), 1(1), 2007.
\end{itemize}

\subsubsection{Further Elaboration:}\label{further-elaboration}

\url{https://docs.google.com/document/d/13haJthH50zWkUDmupgW31NK3xq_aHxab41QQLrzKcTQ/edit}

Since we do not have enough space in this report, further analysis and
interpretation can be accessed via this link.

\subsubsection{Acknowledgments:}\label{acknowledgments}

We extend our heartfelt thanks to Isaiah Katz and Dr.~Uma Ravat. Their
assistance was invaluable, and we couldn't have completed this project
without their support.

\subsection{Appendices}\label{appendices}

\begin{itemize}
\tightlist
\item
  Figures
\end{itemize}

\includegraphics{Report_files/figure-latex/unnamed-chunk-1-1.pdf}

Music

\includegraphics{https://lh7-us.googleusercontent.com/docsz/AD_4nXesONswct9tFSOH6oOdFKIMoGnyUnGckYCNFPEpcqF9hK6ekUv5Y6_tfuz8mV4EGNWVAwUwY6vciG-jG4MzGxlraVLoT7fW95s7uECSLxbMftMbK7kofPjtsu_DIvnX0RmubGtMRyF8kIjdxzaJrm1IUfI?key=h0K1bMW7BkU587zlvlBCtw}

Book

\includegraphics{https://lh7-us.googleusercontent.com/docsz/AD_4nXc2mQS1hMogVBgcRsimXvnm5HyjKObbh0b6iAImCZSMI3OzB2xZeAGQK8916qcT9q6Pp9_2kwSO1Shr4WwOXxrmLfAQ2TqgFgJG_9u-H3kwhhiglvLuFfRfoZoRPW9lqetuwSd6b32h0StDNaf03tCB5xmu?key=h0K1bMW7BkU587zlvlBCtw}

Video

\includegraphics{https://lh7-us.googleusercontent.com/docsz/AD_4nXc0cirA6bxIPKbRBjoY2hbl5J73YZoEf9rlqpuJhvP1FjEM8e722ixP4MbDrIHwkI1zuP3tfpy9saR8kU_204hsusP3ynAG6qVPTUYvib3uO-c_pnIP1Vv3zcMMp5y6_P0Wurs2plwi0_mQMlHIbIh-UmGN?key=h0K1bMW7BkU587zlvlBCtw}

DVD

\includegraphics{https://lh7-us.googleusercontent.com/docsz/AD_4nXews8ePjQRLxghyatuZy0mOtxZbYHykPoXBqIQGqVA3wKNy_qxmN19VqRBLiW2MhQm3vrZjHycVR9Z9t8_2ZvJa9Rbfm7Al_UQsA4pjVrKLiTgUZAUfUbmU4zmSvhrIAR40F90B6UeMGA9RnuWRLVBsZIg?key=h0K1bMW7BkU587zlvlBCtw}

community:

\includegraphics{https://lh7-us.googleusercontent.com/docsz/AD_4nXfSGWgOKeaOI7qzHNWdOtkWYLIQZ4JtEsP8aYBWKj8QTc_O7OBCZrTyH2YhF6jpB9dGDcMti69W7j3GU6DSjx5vZspl-PFcjMGqaJa63hSUJI5dLLEPCpU0KlTBV4aVUk2IAxPM5BzO88zHPv8UAWvJz6Mu?key=h0K1bMW7BkU587zlvlBCtw}

\includegraphics{https://lh7-us.googleusercontent.com/docsz/AD_4nXcM4F-P3aqadZ_ArGZ6wn-qiPnpa2LNbVyOSg_D_-cXuKpcdJOXuly3dgOtkiSSv5T6pmuShK6e_sZudHn-KXNWV-p-LCc2Zbefe4Ni7NzjDEGGa6omwCrIcetHG0bKc4TYiN0oEJ_NIKtXWd3NsKwnEO8?key=h0K1bMW7BkU587zlvlBCtw}

\includegraphics{https://lh7-us.googleusercontent.com/docsz/AD_4nXc5KH2emgCn935pzQkszXzOoU3oUafAKtMYfZRPVK5_eU8tIDrerK8tpE-jmHF3QVTSx_sqDt82KE9mGusfin3_3ir3TMiATzO7597vEfoQgDe-jNVJYR_kG0FBdqzbrJ69pePaNMb--yQiUUSqnOjhLP3l?key=h0K1bMW7BkU587zlvlBCtw}

\includegraphics{https://lh7-us.googleusercontent.com/docsz/AD_4nXeXLvMfzqVvQWRQFZjEvRlQAfGNjbGmrUf5j0TLb_-UdjgZ9xLaeOF8EByD4V11ij6cafa86izszHeDikZKn-s1HCpNtcuXjrYEHFAjPau47I1sh8Q45UkHYKPDYmPqA9VtS-izLcS-BQijJGPotkj_Y9Id?key=h0K1bMW7BkU587zlvlBCtw}

\includegraphics{https://lh7-us.googleusercontent.com/docsz/AD_4nXc5KH2emgCn935pzQkszXzOoU3oUafAKtMYfZRPVK5_eU8tIDrerK8tpE-jmHF3QVTSx_sqDt82KE9mGusfin3_3ir3TMiATzO7597vEfoQgDe-jNVJYR_kG0FBdqzbrJ69pePaNMb--yQiUUSqnOjhLP3l?key=h0K1bMW7BkU587zlvlBCtw}

\includegraphics{https://lh7-us.googleusercontent.com/docsz/AD_4nXfnErLL-IcASsJpoYb6MeCbcXbiiopqCQsPeA8VpvZ7NI1kpnc6RKhepDGJpy4VIIwHASkPmuPY_W4N2hNF_nPd7ZaU7iKxeyAENw-g0I2fVN8sh5FcxYe0I3wmtXvy88FJ6qlIUCduA9tLqO0Mra_wue7g?key=h0K1bMW7BkU587zlvlBCtw}

\includegraphics{https://lh7-us.googleusercontent.com/docsz/AD_4nXeIBaVzL2ksTIAaYHYMeJNf3_M28rqhfEUEDMRfQaiMXscISFk99ABskS4WYWgXt7gcQp7Cz5aAwLuxb8qvsTyye_x4YpPUojb-kA2gyMGP49jmxHfoHQwD-5DvDaM0kn4tG7FP60MwEO3kexSiZbRt0h06?key=h0K1bMW7BkU587zlvlBCtw}

\includegraphics{Report_files/figure-latex/unnamed-chunk-2-1.pdf}
\includegraphics{Report_files/figure-latex/unnamed-chunk-2-2.pdf}
\includegraphics{Report_files/figure-latex/unnamed-chunk-2-3.pdf}
\includegraphics{Report_files/figure-latex/unnamed-chunk-2-4.pdf}

\begin{verbatim}
## [1] 0.3675181
\end{verbatim}

\includegraphics{Report_files/figure-latex/unnamed-chunk-2-5.pdf}
\includegraphics{Report_files/figure-latex/unnamed-chunk-2-6.pdf}
\includegraphics{Report_files/figure-latex/unnamed-chunk-2-7.pdf}

\begin{itemize}
\tightlist
\item
  Code
\end{itemize}

\begin{Shaded}
\begin{Highlighting}[]
\CommentTok{\# Loading Library}
\FunctionTok{library}\NormalTok{(readr)}
\FunctionTok{library}\NormalTok{(igraph)}
\FunctionTok{library}\NormalTok{(rsample)}
\FunctionTok{library}\NormalTok{(dplyr)}
\FunctionTok{library}\NormalTok{(stringr)}
\FunctionTok{library}\NormalTok{(plotly)}

\DocumentationTok{\#\#\#\# Data Cleaning}
\CommentTok{\# read data}
\NormalTok{meta }\OtherTok{\textless{}{-}} \FunctionTok{readLines}\NormalTok{(}\StringTok{\textquotesingle{}../data/amazon{-}meta.txt\textquotesingle{}}\NormalTok{)}

\CommentTok{\# map into df with 5 columns: Id, Title, Group, Categories, Subcategory}
\NormalTok{meta }\OtherTok{\textless{}{-}} \FunctionTok{data.frame}\NormalTok{(}\AttributeTok{line =}\NormalTok{ meta, }\AttributeTok{stringsAsFactors =} \ConstantTok{FALSE}\NormalTok{)}

\NormalTok{filtered\_meta }\OtherTok{\textless{}{-}}\NormalTok{ meta }\SpecialCharTok{\%\textgreater{}\%}
  \FunctionTok{mutate}\NormalTok{(}\AttributeTok{keep =} \FunctionTok{grepl}\NormalTok{(}\StringTok{"Id:|title:|group:|categories:"}\NormalTok{, line) }\SpecialCharTok{|}
                 \FunctionTok{lag}\NormalTok{(}\FunctionTok{grepl}\NormalTok{(}\StringTok{"categories:"}\NormalTok{, line), }\AttributeTok{default =} \ConstantTok{FALSE}\NormalTok{)) }\SpecialCharTok{\%\textgreater{}\%}
  \FunctionTok{filter}\NormalTok{(keep) }\SpecialCharTok{\%\textgreater{}\%}
  \FunctionTok{select}\NormalTok{(}\SpecialCharTok{{-}}\NormalTok{keep)}

\CommentTok{\# map to df}
\NormalTok{result }\OtherTok{\textless{}{-}} \FunctionTok{data.frame}\NormalTok{(}\AttributeTok{Id =} \FunctionTok{character}\NormalTok{(),}
                     \AttributeTok{Title =} \FunctionTok{character}\NormalTok{(),}
                     \AttributeTok{Group =} \FunctionTok{character}\NormalTok{(),}
                     \AttributeTok{Categories =} \FunctionTok{character}\NormalTok{(),}
                     \AttributeTok{Subcategory =} \FunctionTok{character}\NormalTok{(),}
                     \AttributeTok{stringsAsFactors =} \ConstantTok{FALSE}\NormalTok{)}


\CommentTok{\# Process the data}
\NormalTok{i }\OtherTok{\textless{}{-}} \DecValTok{1}
\ControlFlowTok{while}\NormalTok{ (i }\SpecialCharTok{\textless{}=} \FunctionTok{nrow}\NormalTok{(filtered\_meta)) \{}
\NormalTok{  current\_row }\OtherTok{\textless{}{-}}\NormalTok{ filtered\_meta}\SpecialCharTok{$}\NormalTok{line[i]}
  \ControlFlowTok{if}\NormalTok{ (}\FunctionTok{grepl}\NormalTok{(}\StringTok{"Id:  "}\NormalTok{, current\_row)) \{}
    \CommentTok{\# Check if there are at least three more rows and they match the expected titles}
    \ControlFlowTok{if}\NormalTok{ (i }\SpecialCharTok{+} \DecValTok{4} \SpecialCharTok{\textless{}=} \FunctionTok{nrow}\NormalTok{(filtered\_meta) }\SpecialCharTok{\&\&}
        \FunctionTok{grepl}\NormalTok{(}\StringTok{"  title: "}\NormalTok{, filtered\_meta}\SpecialCharTok{$}\NormalTok{line[i }\SpecialCharTok{+} \DecValTok{1}\NormalTok{]) }\SpecialCharTok{\&\&}
        \FunctionTok{grepl}\NormalTok{(}\StringTok{"  group: "}\NormalTok{, filtered\_meta}\SpecialCharTok{$}\NormalTok{line[i }\SpecialCharTok{+} \DecValTok{2}\NormalTok{]) }\SpecialCharTok{\&\&}
        \FunctionTok{grepl}\NormalTok{(}\StringTok{"  categories: "}\NormalTok{, filtered\_meta}\SpecialCharTok{$}\NormalTok{line[i }\SpecialCharTok{+} \DecValTok{3}\NormalTok{])) \{}
      
      \CommentTok{\# Extract and clean the data}
\NormalTok{      id }\OtherTok{\textless{}{-}} \FunctionTok{as.integer}\NormalTok{(}\FunctionTok{sub}\NormalTok{(}\StringTok{"Id:  "}\NormalTok{, }\StringTok{""}\NormalTok{, filtered\_meta}\SpecialCharTok{$}\NormalTok{line[i]))}
\NormalTok{      title }\OtherTok{\textless{}{-}} \FunctionTok{sub}\NormalTok{(}\StringTok{"  title: "}\NormalTok{, }\StringTok{""}\NormalTok{, filtered\_meta}\SpecialCharTok{$}\NormalTok{line[i }\SpecialCharTok{+} \DecValTok{1}\NormalTok{])}
\NormalTok{      group }\OtherTok{\textless{}{-}} \FunctionTok{sub}\NormalTok{(}\StringTok{"  group: "}\NormalTok{, }\StringTok{""}\NormalTok{, filtered\_meta}\SpecialCharTok{$}\NormalTok{line[i }\SpecialCharTok{+} \DecValTok{2}\NormalTok{])}
\NormalTok{      categories }\OtherTok{\textless{}{-}} \FunctionTok{as.integer}\NormalTok{(}\FunctionTok{sub}\NormalTok{(}\StringTok{"  categories: "}\NormalTok{, }\StringTok{""}\NormalTok{, filtered\_meta}\SpecialCharTok{$}\NormalTok{line[i }\SpecialCharTok{+} \DecValTok{3}\NormalTok{]))}
      \ControlFlowTok{if}\NormalTok{ (categories }\SpecialCharTok{==} \DecValTok{0}\NormalTok{) \{}
\NormalTok{        subcategory }\OtherTok{\textless{}{-}} \ConstantTok{NA}  \CommentTok{\# Return NA if categories are 0}
\NormalTok{      \}}
      \ControlFlowTok{else} \ControlFlowTok{if}\NormalTok{ (categories }\SpecialCharTok{\textgreater{}} \DecValTok{0}\NormalTok{)\{}
        \ControlFlowTok{if}\NormalTok{ (group }\SpecialCharTok{\%in\%} \FunctionTok{c}\NormalTok{(}\StringTok{"Music"}\NormalTok{, }\StringTok{"Book"}\NormalTok{, }\StringTok{"Video"}\NormalTok{))\{}
\NormalTok{          sub }\OtherTok{\textless{}{-}}\NormalTok{ filtered\_meta}\SpecialCharTok{$}\NormalTok{line[i }\SpecialCharTok{+} \DecValTok{4}\NormalTok{]}
\NormalTok{          subcategory }\OtherTok{\textless{}{-}} \FunctionTok{str\_split}\NormalTok{(sub, }\StringTok{"}\SpecialCharTok{\textbackslash{}\textbackslash{}}\StringTok{|"}\NormalTok{)[[}\DecValTok{1}\NormalTok{]][}\DecValTok{4}\NormalTok{] }\SpecialCharTok{\%\textgreater{}\%} \FunctionTok{str\_replace\_all}\NormalTok{(}\StringTok{"}\SpecialCharTok{\textbackslash{}\textbackslash{}}\StringTok{[.*?}\SpecialCharTok{\textbackslash{}\textbackslash{}}\StringTok{]"}\NormalTok{, }\StringTok{""}\NormalTok{)}
          \ControlFlowTok{if}\NormalTok{ (subcategory }\SpecialCharTok{\%in\%} \FunctionTok{c}\NormalTok{(}\StringTok{\textquotesingle{}Reference\textquotesingle{}}\NormalTok{, }\StringTok{\textquotesingle{}Genres\textquotesingle{}}\NormalTok{))\{}
\NormalTok{            subcategory }\OtherTok{\textless{}{-}} \FunctionTok{str\_split}\NormalTok{(sub, }\StringTok{"}\SpecialCharTok{\textbackslash{}\textbackslash{}}\StringTok{|"}\NormalTok{)[[}\DecValTok{1}\NormalTok{]][}\DecValTok{5}\NormalTok{] }\SpecialCharTok{\%\textgreater{}\%} \FunctionTok{str\_replace\_all}\NormalTok{(}\StringTok{"}\SpecialCharTok{\textbackslash{}\textbackslash{}}\StringTok{[.*?}\SpecialCharTok{\textbackslash{}\textbackslash{}}\StringTok{]"}\NormalTok{, }\StringTok{""}\NormalTok{)}
\NormalTok{          \}}
\NormalTok{        \}}
        \ControlFlowTok{else} \ControlFlowTok{if}\NormalTok{ (group }\SpecialCharTok{==} \StringTok{"DVD"}\NormalTok{)\{}
\NormalTok{          sub }\OtherTok{\textless{}{-}}\NormalTok{ filtered\_meta}\SpecialCharTok{$}\NormalTok{line[i }\SpecialCharTok{+} \DecValTok{4}\NormalTok{]}
\NormalTok{          subcategory }\OtherTok{\textless{}{-}} \FunctionTok{str\_split}\NormalTok{(sub, }\StringTok{"}\SpecialCharTok{\textbackslash{}\textbackslash{}}\StringTok{|"}\NormalTok{)[[}\DecValTok{1}\NormalTok{]][}\DecValTok{5}\NormalTok{] }\SpecialCharTok{\%\textgreater{}\%} \FunctionTok{str\_replace\_all}\NormalTok{(}\StringTok{"}\SpecialCharTok{\textbackslash{}\textbackslash{}}\StringTok{[.*?}\SpecialCharTok{\textbackslash{}\textbackslash{}}\StringTok{]"}\NormalTok{, }\StringTok{""}\NormalTok{)}
\NormalTok{        \}}
\NormalTok{      \}}
      \CommentTok{\# Append to result dataframe}
\NormalTok{      result }\OtherTok{\textless{}{-}} \FunctionTok{rbind}\NormalTok{(result, }\FunctionTok{data.frame}\NormalTok{(}\AttributeTok{Id =}\NormalTok{ id,}
                                         \AttributeTok{Title =}\NormalTok{ title,}
                                         \AttributeTok{Group =}\NormalTok{ group,}
                                         \AttributeTok{Categories =}\NormalTok{ categories,}
                                         \AttributeTok{Subcategory =}\NormalTok{ subcategory,}
                                         \AttributeTok{stringsAsFactors =} \ConstantTok{FALSE}\NormalTok{))}
      \CommentTok{\# Move index forward by 4}
\NormalTok{      i }\OtherTok{\textless{}{-}}\NormalTok{ i }\SpecialCharTok{+} \DecValTok{5}
\NormalTok{    \}}
    \ControlFlowTok{else}\NormalTok{ \{}
      \CommentTok{\# Skip the current Id and all following rows until next Id or end of data}
\NormalTok{      i }\OtherTok{\textless{}{-}}\NormalTok{ i }\SpecialCharTok{+} \DecValTok{1}
      \ControlFlowTok{while}\NormalTok{(i }\SpecialCharTok{\textless{}=} \FunctionTok{nrow}\NormalTok{(filtered\_meta) }\SpecialCharTok{\&\&} \SpecialCharTok{!}\FunctionTok{grepl}\NormalTok{(}\StringTok{"Id:  "}\NormalTok{, filtered\_meta}\SpecialCharTok{$}\NormalTok{line[i])) \{}
\NormalTok{        i }\OtherTok{\textless{}{-}}\NormalTok{ i }\SpecialCharTok{+} \DecValTok{1}
\NormalTok{      \}}
\NormalTok{    \}}
\NormalTok{  \}}
\NormalTok{\}}

\CommentTok{\# write to csv}
\NormalTok{result }\SpecialCharTok{\%\textgreater{}\%} \FunctionTok{write.csv}\NormalTok{(}\StringTok{\textquotesingle{}../data/meta\_data.csv\textquotesingle{}}\NormalTok{)}

\CommentTok{\# read edges data}
\NormalTok{meta }\OtherTok{=} \FunctionTok{read.csv}\NormalTok{(}\StringTok{\textquotesingle{}../data/meta\_data.csv\textquotesingle{}}\NormalTok{)}
\NormalTok{data }\OtherTok{=} \FunctionTok{read.table}\NormalTok{(}\StringTok{"../data/amazon0601.txt"}\NormalTok{)}
\FunctionTok{colnames}\NormalTok{(data) }\OtherTok{=} \FunctionTok{c}\NormalTok{(}\StringTok{"From"}\NormalTok{, }\StringTok{"To"}\NormalTok{)}

\CommentTok{\# keep edges that link nodes we have info on}
\NormalTok{filtered\_data }\OtherTok{=}\NormalTok{ data[(data}\SpecialCharTok{$}\NormalTok{From }\SpecialCharTok{\%in\%}\NormalTok{ meta}\SpecialCharTok{$}\NormalTok{Id) }\SpecialCharTok{\&}\NormalTok{ (data}\SpecialCharTok{$}\NormalTok{To }\SpecialCharTok{\%in\%}\NormalTok{ meta}\SpecialCharTok{$}\NormalTok{Id), ]}

\CommentTok{\# write to csv}
\NormalTok{filtered\_data }\SpecialCharTok{\%\textgreater{}\%} \FunctionTok{write.csv}\NormalTok{(}\StringTok{\textquotesingle{}../data/filtered\_data.csv\textquotesingle{}}\NormalTok{)}

\CommentTok{\# convert to igraph}
\NormalTok{amz }\OtherTok{\textless{}{-}} \FunctionTok{graph\_from\_data\_frame}\NormalTok{(filtered\_data, }\AttributeTok{directed =} \ConstantTok{TRUE}\NormalTok{)}

\CommentTok{\# create attributes for igraph object from meta data}
\NormalTok{title }\OtherTok{=} \FunctionTok{c}\NormalTok{()}
\NormalTok{group }\OtherTok{=} \FunctionTok{c}\NormalTok{()}
\NormalTok{sub }\OtherTok{=} \FunctionTok{c}\NormalTok{()}

\ControlFlowTok{for}\NormalTok{ (i }\ControlFlowTok{in} \FunctionTok{V}\NormalTok{(amz)}\SpecialCharTok{$}\NormalTok{name)\{}
  \ControlFlowTok{if}\NormalTok{ (}\FunctionTok{as.integer}\NormalTok{(i) }\SpecialCharTok{\%in\%}\NormalTok{ meta}\SpecialCharTok{$}\NormalTok{Id)\{}
\NormalTok{    title }\OtherTok{=} \FunctionTok{append}\NormalTok{(title, meta[meta}\SpecialCharTok{$}\NormalTok{Id }\SpecialCharTok{==} \FunctionTok{as.integer}\NormalTok{(i), ]}\SpecialCharTok{$}\NormalTok{Title)}
\NormalTok{    group }\OtherTok{=} \FunctionTok{append}\NormalTok{(group, meta[meta}\SpecialCharTok{$}\NormalTok{Id }\SpecialCharTok{==} \FunctionTok{as.integer}\NormalTok{(i), ]}\SpecialCharTok{$}\NormalTok{Group)}
\NormalTok{    sub }\OtherTok{=} \FunctionTok{append}\NormalTok{(sub, meta[meta}\SpecialCharTok{$}\NormalTok{Id }\SpecialCharTok{==} \FunctionTok{as.integer}\NormalTok{(i), ]}\SpecialCharTok{$}\NormalTok{Subcategory)}
\NormalTok{  \}}
\NormalTok{\}}

\CommentTok{\# handle missing value}
\NormalTok{sub[}\FunctionTok{is.na}\NormalTok{(sub)] }\OtherTok{\textless{}{-}} \StringTok{"missing"}

\FunctionTok{V}\NormalTok{(amz)}\SpecialCharTok{$}\NormalTok{title }\OtherTok{\textless{}{-}}\NormalTok{ title}
\FunctionTok{V}\NormalTok{(amz)}\SpecialCharTok{$}\NormalTok{group }\OtherTok{\textless{}{-}}\NormalTok{ group}
\FunctionTok{V}\NormalTok{(amz)}\SpecialCharTok{$}\NormalTok{sub }\OtherTok{\textless{}{-}}\NormalTok{ sub}

\CommentTok{\# save igraph object}
\FunctionTok{saveRDS}\NormalTok{(amz, }\AttributeTok{file =} \StringTok{\textquotesingle{}../data/amz\_igraph.rds\textquotesingle{}}\NormalTok{)}

\NormalTok{amz }\OtherTok{\textless{}{-}} \FunctionTok{readRDS}\NormalTok{(}\AttributeTok{file =} \StringTok{\textquotesingle{}../data/amz\_igraph.rds\textquotesingle{}}\NormalTok{)}

\DocumentationTok{\#\#\#\# Sampling Method}
\CommentTok{\# Function to retrieve connected nodes up to a given count}
\NormalTok{retrieve\_connected\_nodes }\OtherTok{\textless{}{-}} \ControlFlowTok{function}\NormalTok{(graph, start\_node, }\AttributeTok{count =} \DecValTok{30}\NormalTok{) \{}
  \CommentTok{\# Initialize the list with the start node}
\NormalTok{  nodes\_to\_explore }\OtherTok{\textless{}{-}} \FunctionTok{list}\NormalTok{(start\_node)}
\NormalTok{  connected\_nodes }\OtherTok{\textless{}{-}} \FunctionTok{c}\NormalTok{(start\_node)}
  
  \CommentTok{\# Keep a list to avoid revisiting nodes}
\NormalTok{  visited\_nodes }\OtherTok{\textless{}{-}} \FunctionTok{numeric}\NormalTok{(}\DecValTok{0}\NormalTok{)}
  
  \CommentTok{\# Explore the graph until we reach the desired number of nodes}
  \ControlFlowTok{while}\NormalTok{ (}\FunctionTok{length}\NormalTok{(nodes\_to\_explore) }\SpecialCharTok{\textgreater{}} \DecValTok{0} \SpecialCharTok{\&\&} \FunctionTok{length}\NormalTok{(connected\_nodes) }\SpecialCharTok{\textless{}}\NormalTok{ count) \{}
\NormalTok{    current\_node }\OtherTok{\textless{}{-}}\NormalTok{ nodes\_to\_explore[[}\DecValTok{1}\NormalTok{]]}
\NormalTok{    nodes\_to\_explore }\OtherTok{\textless{}{-}}\NormalTok{ nodes\_to\_explore[}\SpecialCharTok{{-}}\DecValTok{1}\NormalTok{]  }\CommentTok{\# Remove the explored node}
    
    \CommentTok{\# Skip if already visited}
    \ControlFlowTok{if}\NormalTok{ (current\_node }\SpecialCharTok{\%in\%}\NormalTok{ visited\_nodes) }\ControlFlowTok{next}
    
    \CommentTok{\# Mark as visited}
\NormalTok{    visited\_nodes }\OtherTok{\textless{}{-}} \FunctionTok{c}\NormalTok{(visited\_nodes, current\_node)}
    
    \CommentTok{\# Get neighbors and add to nodes to explore}
\NormalTok{    neighbors }\OtherTok{\textless{}{-}} \FunctionTok{neighbors}\NormalTok{(graph, current\_node)}
\NormalTok{    new\_neighbors }\OtherTok{\textless{}{-}}\NormalTok{ neighbors[}\SpecialCharTok{!}\NormalTok{neighbors }\SpecialCharTok{\%in\%}\NormalTok{ connected\_nodes]}
\NormalTok{    nodes\_to\_explore }\OtherTok{\textless{}{-}} \FunctionTok{c}\NormalTok{(nodes\_to\_explore, }\FunctionTok{as.list}\NormalTok{(new\_neighbors))}
\NormalTok{    connected\_nodes }\OtherTok{\textless{}{-}} \FunctionTok{c}\NormalTok{(connected\_nodes, new\_neighbors)}
    
    \CommentTok{\# Limit the collection if it exceeds the desired count}
    \ControlFlowTok{if}\NormalTok{ (}\FunctionTok{length}\NormalTok{(connected\_nodes) }\SpecialCharTok{\textgreater{}}\NormalTok{ count) \{}
\NormalTok{      connected\_nodes }\OtherTok{\textless{}{-}}\NormalTok{ connected\_nodes[}\DecValTok{1}\SpecialCharTok{:}\NormalTok{count]}
      \ControlFlowTok{break}
\NormalTok{    \}}
\NormalTok{  \}}
  
  \CommentTok{\# Return the vertex sequence of connected nodes}
  \FunctionTok{return}\NormalTok{(connected\_nodes)}
\NormalTok{\}}

\DocumentationTok{\#\#\#\# Visualization}
\CommentTok{\# Define a function to generate and plot the community for a given product group}
\NormalTok{plot\_product\_community }\OtherTok{\textless{}{-}} \ControlFlowTok{function}\NormalTok{(group\_name, main\_title) \{}
  \FunctionTok{set.seed}\NormalTok{(}\DecValTok{194}\NormalTok{)}
  \CommentTok{\# Randomly select one node from the specified group}
\NormalTok{  random\_node }\OtherTok{\textless{}{-}} \FunctionTok{sample}\NormalTok{(}\FunctionTok{V}\NormalTok{(amz)[}\FunctionTok{V}\NormalTok{(amz)}\SpecialCharTok{$}\NormalTok{group }\SpecialCharTok{==}\NormalTok{ group\_name], }\DecValTok{1}\NormalTok{)}
  
  \CommentTok{\# Apply function to get 200 related nodes}
\NormalTok{  nodes }\OtherTok{\textless{}{-}} \FunctionTok{retrieve\_connected\_nodes}\NormalTok{(amz, random\_node)}
\NormalTok{  network }\OtherTok{\textless{}{-}} \FunctionTok{induced\_subgraph}\NormalTok{(amz, nodes)}
  
  \CommentTok{\# Choose a layout that spreads out the nodes more effectively}
\NormalTok{  layout }\OtherTok{\textless{}{-}} \FunctionTok{layout\_with\_fr}\NormalTok{(network)}
  
  \CommentTok{\# Set graph margins to zero}
  \FunctionTok{par}\NormalTok{(}\AttributeTok{mar =} \FunctionTok{c}\NormalTok{(}\DecValTok{0}\NormalTok{, }\DecValTok{0}\NormalTok{, }\DecValTok{2}\NormalTok{, }\DecValTok{0}\NormalTok{))}
  
  \CommentTok{\# Base plot}
  \FunctionTok{plot}\NormalTok{(network, }\AttributeTok{layout =}\NormalTok{ layout,}
       \AttributeTok{vertex.color =} \StringTok{"\#88398A"}\NormalTok{,}
       \AttributeTok{vertex.frame.color =} \StringTok{"\#FFFFFF"}\NormalTok{,}
       \AttributeTok{vertex.size =} \DecValTok{5}\NormalTok{,}
       \AttributeTok{vertex.label =} \FunctionTok{V}\NormalTok{(network)}\SpecialCharTok{$}\NormalTok{name,}
       \AttributeTok{vertex.label.dist =} \DecValTok{1}\NormalTok{,}
       \AttributeTok{vertex.label.cex =} \FloatTok{0.8}\NormalTok{,}
       \AttributeTok{vertex.label.color =} \StringTok{"black"}\NormalTok{,}
       \AttributeTok{vertex.label.font =} \DecValTok{2}\NormalTok{,}
       \AttributeTok{edge.color =} \StringTok{"gray50"}\NormalTok{,}
       \AttributeTok{edge.width =} \FloatTok{0.2}\NormalTok{,}
       \AttributeTok{edge.arrow.size =} \FloatTok{0.1}\NormalTok{,}
       \AttributeTok{main =} \FunctionTok{paste}\NormalTok{(main\_title, }\StringTok{"Base Plot"}\NormalTok{),}
       \AttributeTok{bg =} \StringTok{"white"}
\NormalTok{  )}
  
  \CommentTok{\# Community plot}
\NormalTok{  cluster }\OtherTok{\textless{}{-}} \FunctionTok{cluster\_optimal}\NormalTok{(network)}
\NormalTok{  mycomcols }\OtherTok{\textless{}{-}} \FunctionTok{c}\NormalTok{(}\StringTok{"black"}\NormalTok{, }\StringTok{"\#D3D3D3"}\NormalTok{, }\StringTok{"\#88398A"}\NormalTok{)}
  
  \FunctionTok{plot}\NormalTok{(network, }\AttributeTok{layout =}\NormalTok{ layout,}
       \AttributeTok{vertex.color =}\NormalTok{ mycomcols[cluster}\SpecialCharTok{$}\NormalTok{membership],}
       \AttributeTok{vertex.frame.color =} \StringTok{"\#FFFFFF"}\NormalTok{,}
       \AttributeTok{vertex.size =} \FunctionTok{sqrt}\NormalTok{(}\FunctionTok{degree}\NormalTok{(network)) }\SpecialCharTok{*} \DecValTok{2}\NormalTok{,}
       \AttributeTok{vertex.label =} \FunctionTok{V}\NormalTok{(network)}\SpecialCharTok{$}\NormalTok{name,}
       \AttributeTok{vertex.label.dist =} \DecValTok{1}\NormalTok{,}
       \AttributeTok{vertex.label.cex =} \FloatTok{0.6}\NormalTok{,}
       \AttributeTok{vertex.label.color =} \StringTok{"black"}\NormalTok{,}
       \AttributeTok{vertex.label.font =} \DecValTok{2}\NormalTok{,}
       \AttributeTok{edge.color =} \StringTok{"gray50"}\NormalTok{,}
       \AttributeTok{edge.width =} \FloatTok{0.2}\NormalTok{,}
       \AttributeTok{edge.arrow.size =} \FloatTok{0.1}\NormalTok{,}
       \AttributeTok{main =} \FunctionTok{paste}\NormalTok{(main\_title, }\StringTok{"Community"}\NormalTok{),}
       \AttributeTok{bg =} \StringTok{"white"}
\NormalTok{  )}
  
  \CommentTok{\# Get vertex data including the degree for size scaling}
\NormalTok{  vertex\_data }\OtherTok{\textless{}{-}} \FunctionTok{data.frame}\NormalTok{(}
    \AttributeTok{Id =} \FunctionTok{V}\NormalTok{(network)}\SpecialCharTok{$}\NormalTok{name,}
    \AttributeTok{x =}\NormalTok{ layout[, }\DecValTok{1}\NormalTok{],}
    \AttributeTok{y =}\NormalTok{ layout[, }\DecValTok{2}\NormalTok{],}
    \AttributeTok{degree =} \FunctionTok{degree}\NormalTok{(network),}
    \AttributeTok{Title =} \FunctionTok{V}\NormalTok{(network)}\SpecialCharTok{$}\NormalTok{title,}
    \AttributeTok{Group =} \FunctionTok{V}\NormalTok{(network)}\SpecialCharTok{$}\NormalTok{group,}
    \AttributeTok{Category =} \FunctionTok{V}\NormalTok{(network)}\SpecialCharTok{$}\NormalTok{sub}
\NormalTok{  )}
  
  \CommentTok{\# Enhance hover info by including all attributes except x, y coordinates}
\NormalTok{  vertex\_data}\SpecialCharTok{$}\NormalTok{hoverinfo }\OtherTok{\textless{}{-}} \FunctionTok{apply}\NormalTok{(vertex\_data[, }\SpecialCharTok{{-}}\FunctionTok{c}\NormalTok{(}\DecValTok{2}\NormalTok{, }\DecValTok{3}\NormalTok{)], }\DecValTok{1}\NormalTok{, }\ControlFlowTok{function}\NormalTok{(row) \{}
    \FunctionTok{paste}\NormalTok{(}\FunctionTok{names}\NormalTok{(row), row, }\AttributeTok{sep=}\StringTok{": "}\NormalTok{, }\AttributeTok{collapse=}\StringTok{"\textless{}br\textgreater{}"}\NormalTok{)}
\NormalTok{  \})}
  
  \CommentTok{\# Get edge data}
\NormalTok{  edge\_data }\OtherTok{\textless{}{-}} \FunctionTok{get.data.frame}\NormalTok{(network, }\AttributeTok{what =} \StringTok{"edges"}\NormalTok{)}
  
  \CommentTok{\# Join edge data with vertex data to get coordinates for \textquotesingle{}from\textquotesingle{} and \textquotesingle{}to\textquotesingle{}}
\NormalTok{  edge\_data }\OtherTok{\textless{}{-}} \FunctionTok{merge}\NormalTok{(edge\_data, vertex\_data, }\AttributeTok{by.x =} \StringTok{"from"}\NormalTok{, }\AttributeTok{by.y =} \StringTok{"Id"}\NormalTok{, }\AttributeTok{all.x =} \ConstantTok{TRUE}\NormalTok{)}
\NormalTok{  edge\_data }\OtherTok{\textless{}{-}} \FunctionTok{merge}\NormalTok{(edge\_data, vertex\_data, }\AttributeTok{by.x =} \StringTok{"to"}\NormalTok{, }\AttributeTok{by.y =} \StringTok{"Id"}\NormalTok{, }\AttributeTok{all.x =} \ConstantTok{TRUE}\NormalTok{, }\AttributeTok{suffixes =} \FunctionTok{c}\NormalTok{(}\StringTok{".from"}\NormalTok{, }\StringTok{".to"}\NormalTok{))}
  
  \CommentTok{\# Prepare data for Plotly plot}
\NormalTok{  edges }\OtherTok{\textless{}{-}} \FunctionTok{list}\NormalTok{(}
    \AttributeTok{x =} \FunctionTok{c}\NormalTok{(}\FunctionTok{rbind}\NormalTok{(edge\_data}\SpecialCharTok{$}\NormalTok{x.from, edge\_data}\SpecialCharTok{$}\NormalTok{x.to, }\ConstantTok{NA}\NormalTok{)),}
    \AttributeTok{y =} \FunctionTok{c}\NormalTok{(}\FunctionTok{rbind}\NormalTok{(edge\_data}\SpecialCharTok{$}\NormalTok{y.from, edge\_data}\SpecialCharTok{$}\NormalTok{y.to, }\ConstantTok{NA}\NormalTok{)),}
    \AttributeTok{type =} \StringTok{"scatter"}\NormalTok{,}
    \AttributeTok{mode =} \StringTok{"lines"}\NormalTok{,}
    \AttributeTok{line =} \FunctionTok{list}\NormalTok{(}\AttributeTok{color =} \StringTok{"grey"}\NormalTok{, }\AttributeTok{width =} \FloatTok{0.5}\NormalTok{)}
\NormalTok{  )}
  
\NormalTok{  nodes }\OtherTok{\textless{}{-}} \FunctionTok{list}\NormalTok{(}
    \AttributeTok{x =}\NormalTok{ vertex\_data}\SpecialCharTok{$}\NormalTok{x,}
    \AttributeTok{y =}\NormalTok{ vertex\_data}\SpecialCharTok{$}\NormalTok{y,}
    \AttributeTok{hovertext =}\NormalTok{ vertex\_data}\SpecialCharTok{$}\NormalTok{hoverinfo,}
    \AttributeTok{mode =} \StringTok{"markers"}\NormalTok{,}
    \AttributeTok{marker =} \FunctionTok{list}\NormalTok{(}\AttributeTok{size =}\NormalTok{ vertex\_data}\SpecialCharTok{$}\NormalTok{degree }\SpecialCharTok{*} \DecValTok{2}\NormalTok{,}
                  \AttributeTok{color =}\NormalTok{ mycomcols[cluster}\SpecialCharTok{$}\NormalTok{membership]),}
    \AttributeTok{type =} \StringTok{"scatter"}\NormalTok{,}
    \AttributeTok{hoverinfo =} \StringTok{"text"}
\NormalTok{  )}
  
  \CommentTok{\# Create the plot}
  \FunctionTok{plot\_ly}\NormalTok{() }\SpecialCharTok{\%\textgreater{}\%}
    \FunctionTok{add\_trace}\NormalTok{(}\AttributeTok{x =}\NormalTok{ edges}\SpecialCharTok{$}\NormalTok{x, }\AttributeTok{y =}\NormalTok{ edges}\SpecialCharTok{$}\NormalTok{y, }\AttributeTok{mode =}\NormalTok{ edges}\SpecialCharTok{$}\NormalTok{mode, }\AttributeTok{type =}\NormalTok{ edges}\SpecialCharTok{$}\NormalTok{type, }\AttributeTok{line =}\NormalTok{ edges}\SpecialCharTok{$}\NormalTok{line) }\SpecialCharTok{\%\textgreater{}\%}
    \FunctionTok{add\_trace}\NormalTok{(}\AttributeTok{x =}\NormalTok{ nodes}\SpecialCharTok{$}\NormalTok{x, }\AttributeTok{y =}\NormalTok{ nodes}\SpecialCharTok{$}\NormalTok{y, }\AttributeTok{hovertext =}\NormalTok{ nodes}\SpecialCharTok{$}\NormalTok{hovertext, }\AttributeTok{mode =}\NormalTok{ nodes}\SpecialCharTok{$}\NormalTok{mode, }\AttributeTok{type =}\NormalTok{ nodes}\SpecialCharTok{$}\NormalTok{type, }\AttributeTok{hoverinfo =} \StringTok{"text"}\NormalTok{, }\AttributeTok{marker =}\NormalTok{ nodes}\SpecialCharTok{$}\NormalTok{marker) }\SpecialCharTok{\%\textgreater{}\%}
    \FunctionTok{layout}\NormalTok{(}
      \AttributeTok{title =} \FunctionTok{paste}\NormalTok{(}\StringTok{"Network Visualization of"}\NormalTok{, main\_title),}
      \AttributeTok{xaxis =} \FunctionTok{list}\NormalTok{(}\AttributeTok{showgrid =} \ConstantTok{FALSE}\NormalTok{, }\AttributeTok{zeroline =} \ConstantTok{FALSE}\NormalTok{, }\AttributeTok{showticklabels =} \ConstantTok{FALSE}\NormalTok{),}
      \AttributeTok{yaxis =} \FunctionTok{list}\NormalTok{(}\AttributeTok{showgrid =} \ConstantTok{FALSE}\NormalTok{, }\AttributeTok{zeroline =} \ConstantTok{FALSE}\NormalTok{, }\AttributeTok{showticklabels =} \ConstantTok{FALSE}\NormalTok{),}
      \AttributeTok{hovermode =} \StringTok{\textquotesingle{}closest\textquotesingle{}}
\NormalTok{    )}
\NormalTok{\}}

\CommentTok{\# Music group}
\FunctionTok{plot\_product\_community}\NormalTok{(}\StringTok{"Music"}\NormalTok{, }\StringTok{"Music Products"}\NormalTok{)}

\CommentTok{\# Book group}
\FunctionTok{plot\_product\_community}\NormalTok{(}\StringTok{"Book"}\NormalTok{, }\StringTok{"Book Products"}\NormalTok{)}

\CommentTok{\# Video group}
\FunctionTok{plot\_product\_community}\NormalTok{(}\StringTok{"Video"}\NormalTok{, }\StringTok{"Video Products"}\NormalTok{)}

\CommentTok{\# DVD group}
\FunctionTok{plot\_product\_community}\NormalTok{(}\StringTok{"DVD"}\NormalTok{, }\StringTok{"DVD Products"}\NormalTok{)}

\DocumentationTok{\#\#\#\# Network Metrics}
\DocumentationTok{\#\# Music Products}
\CommentTok{\# Density}
\NormalTok{network\_density }\OtherTok{\textless{}{-}} \FunctionTok{edge\_density}\NormalTok{(music.network)}
\FunctionTok{cat}\NormalTok{(}\StringTok{"Network Density:"}\NormalTok{, network\_density, }\StringTok{"}\SpecialCharTok{\textbackslash{}n}\StringTok{"}\NormalTok{)}

\CommentTok{\# Average Path Length}
\NormalTok{avg\_path\_length }\OtherTok{\textless{}{-}} \FunctionTok{average.path.length}\NormalTok{(music.network, }\AttributeTok{directed =} \ConstantTok{FALSE}\NormalTok{)}
\FunctionTok{cat}\NormalTok{(}\StringTok{"Average Path Length:"}\NormalTok{, avg\_path\_length, }\StringTok{"}\SpecialCharTok{\textbackslash{}n}\StringTok{"}\NormalTok{)}

\CommentTok{\# Diameter}
\NormalTok{network\_diameter }\OtherTok{\textless{}{-}} \FunctionTok{diameter}\NormalTok{(music.network, }\AttributeTok{directed =} \ConstantTok{FALSE}\NormalTok{)}
\FunctionTok{cat}\NormalTok{(}\StringTok{"Network Diameter:"}\NormalTok{, network\_diameter, }\StringTok{"}\SpecialCharTok{\textbackslash{}n}\StringTok{"}\NormalTok{)}

\CommentTok{\# Node Metrics}
\CommentTok{\# Degree}
\NormalTok{node\_degree }\OtherTok{\textless{}{-}} \FunctionTok{degree}\NormalTok{(music.network)}
\FunctionTok{cat}\NormalTok{(}\StringTok{"Node Degree:}\SpecialCharTok{\textbackslash{}n}\StringTok{"}\NormalTok{)}
\FunctionTok{print}\NormalTok{(}\FunctionTok{summary}\NormalTok{(node\_degree))}

\CommentTok{\# Betweenness Centrality}
\NormalTok{node\_betweenness }\OtherTok{\textless{}{-}} \FunctionTok{betweenness}\NormalTok{(music.network)}
\FunctionTok{cat}\NormalTok{(}\StringTok{"Node Betweenness Centrality:}\SpecialCharTok{\textbackslash{}n}\StringTok{"}\NormalTok{)}
\FunctionTok{print}\NormalTok{(}\FunctionTok{summary}\NormalTok{(node\_betweenness))}

\CommentTok{\# Closeness Centrality}
\NormalTok{node\_closeness }\OtherTok{\textless{}{-}} \FunctionTok{closeness}\NormalTok{(music.network)}
\FunctionTok{cat}\NormalTok{(}\StringTok{"Node Closeness Centrality:}\SpecialCharTok{\textbackslash{}n}\StringTok{"}\NormalTok{)}
\FunctionTok{print}\NormalTok{(}\FunctionTok{summary}\NormalTok{(node\_closeness))}

\CommentTok{\# Eigenvector Centrality}
\NormalTok{node\_eigenvector }\OtherTok{\textless{}{-}} \FunctionTok{evcent}\NormalTok{(music.network)}\SpecialCharTok{$}\NormalTok{vector}
\FunctionTok{cat}\NormalTok{(}\StringTok{"Node Eigenvector Centrality:}\SpecialCharTok{\textbackslash{}n}\StringTok{"}\NormalTok{)}
\FunctionTok{print}\NormalTok{(}\FunctionTok{summary}\NormalTok{(node\_eigenvector))}

\CommentTok{\# Edge Metrics}
\CommentTok{\# Edge Betweenness}
\NormalTok{edge\_betweenness }\OtherTok{\textless{}{-}} \FunctionTok{edge.betweenness}\NormalTok{(music.network)}
\FunctionTok{cat}\NormalTok{(}\StringTok{"Edge Betweenness:}\SpecialCharTok{\textbackslash{}n}\StringTok{"}\NormalTok{)}
\FunctionTok{print}\NormalTok{(}\FunctionTok{summary}\NormalTok{(edge\_betweenness))}

\DocumentationTok{\#\# Book Products}
\CommentTok{\# Density}
\NormalTok{network\_density }\OtherTok{\textless{}{-}} \FunctionTok{edge\_density}\NormalTok{(book.network)}
\FunctionTok{cat}\NormalTok{(}\StringTok{"Network Density:"}\NormalTok{, network\_density, }\StringTok{"}\SpecialCharTok{\textbackslash{}n}\StringTok{"}\NormalTok{)}

\CommentTok{\# Average Path Length}
\NormalTok{avg\_path\_length }\OtherTok{\textless{}{-}} \FunctionTok{average.path.length}\NormalTok{(book.network, }\AttributeTok{directed =} \ConstantTok{FALSE}\NormalTok{)}
\FunctionTok{cat}\NormalTok{(}\StringTok{"Average Path Length:"}\NormalTok{, avg\_path\_length, }\StringTok{"}\SpecialCharTok{\textbackslash{}n}\StringTok{"}\NormalTok{)}

\CommentTok{\# Diameter}
\NormalTok{network\_diameter }\OtherTok{\textless{}{-}} \FunctionTok{diameter}\NormalTok{(book.network, }\AttributeTok{directed =} \ConstantTok{FALSE}\NormalTok{)}
\FunctionTok{cat}\NormalTok{(}\StringTok{"Network Diameter:"}\NormalTok{, network\_diameter, }\StringTok{"}\SpecialCharTok{\textbackslash{}n}\StringTok{"}\NormalTok{)}

\CommentTok{\# Node Metrics}
\CommentTok{\# Degree}
\NormalTok{node\_degree }\OtherTok{\textless{}{-}} \FunctionTok{degree}\NormalTok{(book.network)}
\FunctionTok{cat}\NormalTok{(}\StringTok{"Node Degree:}\SpecialCharTok{\textbackslash{}n}\StringTok{"}\NormalTok{)}
\FunctionTok{print}\NormalTok{(}\FunctionTok{summary}\NormalTok{(node\_degree))}

\CommentTok{\# Betweenness Centrality}
\NormalTok{node\_betweenness }\OtherTok{\textless{}{-}} \FunctionTok{betweenness}\NormalTok{(book.network)}
\FunctionTok{cat}\NormalTok{(}\StringTok{"Node Betweenness Centrality:}\SpecialCharTok{\textbackslash{}n}\StringTok{"}\NormalTok{)}
\FunctionTok{print}\NormalTok{(}\FunctionTok{summary}\NormalTok{(node\_betweenness))}

\CommentTok{\# Closeness Centrality}
\NormalTok{node\_closeness }\OtherTok{\textless{}{-}} \FunctionTok{closeness}\NormalTok{(book.network)}
\FunctionTok{cat}\NormalTok{(}\StringTok{"Node Closeness Centrality:}\SpecialCharTok{\textbackslash{}n}\StringTok{"}\NormalTok{)}
\FunctionTok{print}\NormalTok{(}\FunctionTok{summary}\NormalTok{(node\_closeness))}

\CommentTok{\# Eigenvector Centrality}
\NormalTok{node\_eigenvector }\OtherTok{\textless{}{-}} \FunctionTok{evcent}\NormalTok{(book.network)}\SpecialCharTok{$}\NormalTok{vector}
\FunctionTok{cat}\NormalTok{(}\StringTok{"Node Eigenvector Centrality:}\SpecialCharTok{\textbackslash{}n}\StringTok{"}\NormalTok{)}
\FunctionTok{print}\NormalTok{(}\FunctionTok{summary}\NormalTok{(node\_eigenvector))}

\CommentTok{\# Edge Metrics}
\CommentTok{\# Edge Betweenness}
\NormalTok{edge\_betweenness }\OtherTok{\textless{}{-}} \FunctionTok{edge.betweenness}\NormalTok{(book.network)}
\FunctionTok{cat}\NormalTok{(}\StringTok{"Edge Betweenness:}\SpecialCharTok{\textbackslash{}n}\StringTok{"}\NormalTok{)}
\FunctionTok{print}\NormalTok{(}\FunctionTok{summary}\NormalTok{(edge\_betweenness))}

\DocumentationTok{\#\# Video Products}
\CommentTok{\# Network Metrics}
\CommentTok{\# Density}
\NormalTok{network\_density }\OtherTok{\textless{}{-}} \FunctionTok{edge\_density}\NormalTok{(video.network)}
\FunctionTok{cat}\NormalTok{(}\StringTok{"Network Density:"}\NormalTok{, network\_density, }\StringTok{"}\SpecialCharTok{\textbackslash{}n}\StringTok{"}\NormalTok{)}

\CommentTok{\# Average Path Length}
\NormalTok{avg\_path\_length }\OtherTok{\textless{}{-}} \FunctionTok{average.path.length}\NormalTok{(video.network, }\AttributeTok{directed =} \ConstantTok{FALSE}\NormalTok{)}
\FunctionTok{cat}\NormalTok{(}\StringTok{"Average Path Length:"}\NormalTok{, avg\_path\_length, }\StringTok{"}\SpecialCharTok{\textbackslash{}n}\StringTok{"}\NormalTok{)}

\CommentTok{\# Diameter}
\NormalTok{network\_diameter }\OtherTok{\textless{}{-}} \FunctionTok{diameter}\NormalTok{(video.network, }\AttributeTok{directed =} \ConstantTok{FALSE}\NormalTok{)}
\FunctionTok{cat}\NormalTok{(}\StringTok{"Network Diameter:"}\NormalTok{, network\_diameter, }\StringTok{"}\SpecialCharTok{\textbackslash{}n}\StringTok{"}\NormalTok{)}

\CommentTok{\# Node Metrics}
\CommentTok{\# Degree}
\NormalTok{node\_degree }\OtherTok{\textless{}{-}} \FunctionTok{degree}\NormalTok{(video.network)}
\FunctionTok{cat}\NormalTok{(}\StringTok{"Node Degree:}\SpecialCharTok{\textbackslash{}n}\StringTok{"}\NormalTok{)}
\FunctionTok{print}\NormalTok{(}\FunctionTok{summary}\NormalTok{(node\_degree))}

\CommentTok{\# Betweenness Centrality}
\NormalTok{node\_betweenness }\OtherTok{\textless{}{-}} \FunctionTok{betweenness}\NormalTok{(video.network)}
\FunctionTok{cat}\NormalTok{(}\StringTok{"Node Betweenness Centrality:}\SpecialCharTok{\textbackslash{}n}\StringTok{"}\NormalTok{)}
\FunctionTok{print}\NormalTok{(}\FunctionTok{summary}\NormalTok{(node\_betweenness))}

\CommentTok{\# Closeness Centrality}
\NormalTok{node\_closeness }\OtherTok{\textless{}{-}} \FunctionTok{closeness}\NormalTok{(video.network)}
\FunctionTok{cat}\NormalTok{(}\StringTok{"Node Closeness Centrality:}\SpecialCharTok{\textbackslash{}n}\StringTok{"}\NormalTok{)}
\FunctionTok{print}\NormalTok{(}\FunctionTok{summary}\NormalTok{(node\_closeness))}

\CommentTok{\# Eigenvector Centrality}
\NormalTok{node\_eigenvector }\OtherTok{\textless{}{-}} \FunctionTok{evcent}\NormalTok{(video.network)}\SpecialCharTok{$}\NormalTok{vector}
\FunctionTok{cat}\NormalTok{(}\StringTok{"Node Eigenvector Centrality:}\SpecialCharTok{\textbackslash{}n}\StringTok{"}\NormalTok{)}
\FunctionTok{print}\NormalTok{(}\FunctionTok{summary}\NormalTok{(node\_eigenvector))}

\CommentTok{\# Edge Metrics}
\CommentTok{\# Edge Betweenness}
\NormalTok{edge\_betweenness }\OtherTok{\textless{}{-}} \FunctionTok{edge.betweenness}\NormalTok{(video.network)}
\FunctionTok{cat}\NormalTok{(}\StringTok{"Edge Betweenness:}\SpecialCharTok{\textbackslash{}n}\StringTok{"}\NormalTok{)}
\FunctionTok{print}\NormalTok{(}\FunctionTok{summary}\NormalTok{(edge\_betweenness))}

\DocumentationTok{\#\# DVD Products}
\NormalTok{network\_density }\OtherTok{\textless{}{-}} \FunctionTok{edge\_density}\NormalTok{(dvd.network)}
\FunctionTok{cat}\NormalTok{(}\StringTok{"Network Density:"}\NormalTok{, network\_density, }\StringTok{"}\SpecialCharTok{\textbackslash{}n}\StringTok{"}\NormalTok{)}

\CommentTok{\# Average Path Length}
\NormalTok{avg\_path\_length }\OtherTok{\textless{}{-}} \FunctionTok{average.path.length}\NormalTok{(dvd.network, }\AttributeTok{directed =} \ConstantTok{FALSE}\NormalTok{)}
\FunctionTok{cat}\NormalTok{(}\StringTok{"Average Path Length:"}\NormalTok{, avg\_path\_length, }\StringTok{"}\SpecialCharTok{\textbackslash{}n}\StringTok{"}\NormalTok{)}

\CommentTok{\# Diameter}
\NormalTok{network\_diameter }\OtherTok{\textless{}{-}} \FunctionTok{diameter}\NormalTok{(dvd.network, }\AttributeTok{directed =} \ConstantTok{FALSE}\NormalTok{)}
\FunctionTok{cat}\NormalTok{(}\StringTok{"Network Diameter:"}\NormalTok{, network\_diameter, }\StringTok{"}\SpecialCharTok{\textbackslash{}n}\StringTok{"}\NormalTok{)}

\CommentTok{\# Node Metrics}
\CommentTok{\# Degree}
\NormalTok{node\_degree }\OtherTok{\textless{}{-}} \FunctionTok{degree}\NormalTok{(dvd.network)}
\FunctionTok{cat}\NormalTok{(}\StringTok{"Node Degree:}\SpecialCharTok{\textbackslash{}n}\StringTok{"}\NormalTok{)}
\FunctionTok{print}\NormalTok{(}\FunctionTok{summary}\NormalTok{(node\_degree))}

\CommentTok{\# Betweenness Centrality}
\NormalTok{node\_betweenness }\OtherTok{\textless{}{-}} \FunctionTok{betweenness}\NormalTok{(dvd.network)}
\FunctionTok{cat}\NormalTok{(}\StringTok{"Node Betweenness Centrality:}\SpecialCharTok{\textbackslash{}n}\StringTok{"}\NormalTok{)}
\FunctionTok{print}\NormalTok{(}\FunctionTok{summary}\NormalTok{(node\_betweenness))}

\CommentTok{\# Closeness Centrality}
\NormalTok{node\_closeness }\OtherTok{\textless{}{-}} \FunctionTok{closeness}\NormalTok{(dvd.network)}
\FunctionTok{cat}\NormalTok{(}\StringTok{"Node Closeness Centrality:}\SpecialCharTok{\textbackslash{}n}\StringTok{"}\NormalTok{)}
\FunctionTok{print}\NormalTok{(}\FunctionTok{summary}\NormalTok{(node\_closeness))}

\CommentTok{\# Eigenvector Centrality}
\NormalTok{node\_eigenvector }\OtherTok{\textless{}{-}} \FunctionTok{evcent}\NormalTok{(dvd.network)}\SpecialCharTok{$}\NormalTok{vector}
\FunctionTok{cat}\NormalTok{(}\StringTok{"Node Eigenvector Centrality:}\SpecialCharTok{\textbackslash{}n}\StringTok{"}\NormalTok{)}
\FunctionTok{print}\NormalTok{(}\FunctionTok{summary}\NormalTok{(node\_eigenvector))}

\CommentTok{\# Edge Metrics}
\CommentTok{\# Edge Betweenness}
\NormalTok{edge\_betweenness }\OtherTok{\textless{}{-}} \FunctionTok{edge.betweenness}\NormalTok{(dvd.network)}
\FunctionTok{cat}\NormalTok{(}\StringTok{"Edge Betweenness:}\SpecialCharTok{\textbackslash{}n}\StringTok{"}\NormalTok{)}
\FunctionTok{print}\NormalTok{(}\FunctionTok{summary}\NormalTok{(edge\_betweenness))}

\DocumentationTok{\#\#\#\# Community Detection}
\DocumentationTok{\#\# Music}
\CommentTok{\# Walktrap Algorithm}
\NormalTok{walktrap\_communities }\OtherTok{\textless{}{-}} \FunctionTok{cluster\_walktrap}\NormalTok{(music.network)}
\FunctionTok{cat}\NormalTok{(}\StringTok{"Walktrap Algorithm:}\SpecialCharTok{\textbackslash{}n}\StringTok{"}\NormalTok{)}
\FunctionTok{print}\NormalTok{(}\FunctionTok{membership}\NormalTok{(walktrap\_communities))}
\FunctionTok{cat}\NormalTok{(}\StringTok{"Modularity:"}\NormalTok{, }\FunctionTok{modularity}\NormalTok{(walktrap\_communities), }\StringTok{"}\SpecialCharTok{\textbackslash{}n}\StringTok{"}\NormalTok{)}

\CommentTok{\# Infomap Algorithm}
\NormalTok{infomap\_communities }\OtherTok{\textless{}{-}} \FunctionTok{cluster\_infomap}\NormalTok{(music.network)}
\FunctionTok{cat}\NormalTok{(}\StringTok{"Infomap Algorithm:}\SpecialCharTok{\textbackslash{}n}\StringTok{"}\NormalTok{)}
\FunctionTok{print}\NormalTok{(}\FunctionTok{membership}\NormalTok{(infomap\_communities))}
\FunctionTok{cat}\NormalTok{(}\StringTok{"Modularity:"}\NormalTok{, }\FunctionTok{modularity}\NormalTok{(infomap\_communities), }\StringTok{"}\SpecialCharTok{\textbackslash{}n}\StringTok{"}\NormalTok{)}

\CommentTok{\# Visualize communities }
\FunctionTok{par}\NormalTok{(}\AttributeTok{mfrow=}\FunctionTok{c}\NormalTok{(}\DecValTok{1}\NormalTok{,}\DecValTok{3}\NormalTok{))}
\FunctionTok{plot}\NormalTok{(walktrap\_communities, music.network, }\AttributeTok{main=}\StringTok{"Walktrap Algorithm"}\NormalTok{, }\AttributeTok{vertex.size=}\DecValTok{10}\NormalTok{, }\AttributeTok{edge.arrow.size=}\FloatTok{0.05}\NormalTok{, }\AttributeTok{vertex.label.cex=}\FloatTok{0.7}\NormalTok{, }\AttributeTok{vertex.label.color=}\StringTok{"black"}\NormalTok{)}
\FunctionTok{plot}\NormalTok{(infomap\_communities, music.network, }\AttributeTok{main=}\StringTok{"Infomap Algorithm"}\NormalTok{, }\AttributeTok{vertex.size=}\DecValTok{10}\NormalTok{, }\AttributeTok{edge.arrow.size=}\FloatTok{0.05}\NormalTok{, }\AttributeTok{vertex.label.cex=}\FloatTok{0.7}\NormalTok{, }\AttributeTok{vertex.label.color=}\StringTok{"black"}\NormalTok{)}

\DocumentationTok{\#\# Book}
\CommentTok{\# Walktrap Algorithm}
\NormalTok{walktrap\_communities }\OtherTok{\textless{}{-}} \FunctionTok{cluster\_walktrap}\NormalTok{(book.network)}
\FunctionTok{cat}\NormalTok{(}\StringTok{"Walktrap Algorithm:}\SpecialCharTok{\textbackslash{}n}\StringTok{"}\NormalTok{)}
\FunctionTok{print}\NormalTok{(}\FunctionTok{membership}\NormalTok{(walktrap\_communities))}
\FunctionTok{cat}\NormalTok{(}\StringTok{"Modularity:"}\NormalTok{, }\FunctionTok{modularity}\NormalTok{(walktrap\_communities), }\StringTok{"}\SpecialCharTok{\textbackslash{}n}\StringTok{"}\NormalTok{)}

\CommentTok{\# Infomap Algorithm}
\NormalTok{infomap\_communities }\OtherTok{\textless{}{-}} \FunctionTok{cluster\_infomap}\NormalTok{(book.network)}
\FunctionTok{cat}\NormalTok{(}\StringTok{"Infomap Algorithm:}\SpecialCharTok{\textbackslash{}n}\StringTok{"}\NormalTok{)}
\FunctionTok{print}\NormalTok{(}\FunctionTok{membership}\NormalTok{(infomap\_communities))}
\FunctionTok{cat}\NormalTok{(}\StringTok{"Modularity:"}\NormalTok{, }\FunctionTok{modularity}\NormalTok{(infomap\_communities), }\StringTok{"}\SpecialCharTok{\textbackslash{}n}\StringTok{"}\NormalTok{)}

\CommentTok{\# Visualize communities}
\FunctionTok{par}\NormalTok{(}\AttributeTok{mfrow=}\FunctionTok{c}\NormalTok{(}\DecValTok{1}\NormalTok{,}\DecValTok{2}\NormalTok{))}
\FunctionTok{plot}\NormalTok{(walktrap\_communities, book.network, }\AttributeTok{main=}\StringTok{"Walktrap Algorithm"}\NormalTok{, }\AttributeTok{vertex.size=}\DecValTok{10}\NormalTok{, }\AttributeTok{edge.arrow.size=}\FloatTok{0.05}\NormalTok{, }\AttributeTok{vertex.label.cex=}\FloatTok{0.7}\NormalTok{, }\AttributeTok{vertex.label.color=}\StringTok{"black"}\NormalTok{)}
\FunctionTok{plot}\NormalTok{(infomap\_communities, book.network, }\AttributeTok{main=}\StringTok{"Infomap Algorithm"}\NormalTok{, }\AttributeTok{vertex.size=}\DecValTok{10}\NormalTok{, }\AttributeTok{edge.arrow.size=}\FloatTok{0.05}\NormalTok{, }\AttributeTok{vertex.label.cex=}\FloatTok{0.7}\NormalTok{, }\AttributeTok{vertex.label.color=}\StringTok{"black"}\NormalTok{)}

\DocumentationTok{\#\# Video}
\CommentTok{\# Walktrap Algorithm}
\NormalTok{walktrap\_communities }\OtherTok{\textless{}{-}} \FunctionTok{cluster\_walktrap}\NormalTok{(video.network)}
\FunctionTok{cat}\NormalTok{(}\StringTok{"Walktrap Algorithm:}\SpecialCharTok{\textbackslash{}n}\StringTok{"}\NormalTok{)}
\FunctionTok{print}\NormalTok{(}\FunctionTok{membership}\NormalTok{(walktrap\_communities))}
\FunctionTok{cat}\NormalTok{(}\StringTok{"Modularity:"}\NormalTok{, }\FunctionTok{modularity}\NormalTok{(walktrap\_communities), }\StringTok{"}\SpecialCharTok{\textbackslash{}n}\StringTok{"}\NormalTok{)}

\CommentTok{\# Infomap Algorithm}
\NormalTok{infomap\_communities }\OtherTok{\textless{}{-}} \FunctionTok{cluster\_infomap}\NormalTok{(video.network)}
\FunctionTok{cat}\NormalTok{(}\StringTok{"Infomap Algorithm:}\SpecialCharTok{\textbackslash{}n}\StringTok{"}\NormalTok{)}
\FunctionTok{print}\NormalTok{(}\FunctionTok{membership}\NormalTok{(infomap\_communities))}
\FunctionTok{cat}\NormalTok{(}\StringTok{"Modularity:"}\NormalTok{, }\FunctionTok{modularity}\NormalTok{(infomap\_communities), }\StringTok{"}\SpecialCharTok{\textbackslash{}n}\StringTok{"}\NormalTok{)}

\CommentTok{\# Visualize communities }
\FunctionTok{par}\NormalTok{(}\AttributeTok{mfrow=}\FunctionTok{c}\NormalTok{(}\DecValTok{1}\NormalTok{,}\DecValTok{3}\NormalTok{))}
\FunctionTok{plot}\NormalTok{(walktrap\_communities, video.network, }\AttributeTok{main=}\StringTok{"Walktrap Algorithm"}\NormalTok{, }\AttributeTok{vertex.size=}\DecValTok{10}\NormalTok{, }\AttributeTok{edge.arrow.size=}\FloatTok{0.05}\NormalTok{, }\AttributeTok{vertex.label.cex=}\FloatTok{0.7}\NormalTok{, }\AttributeTok{vertex.label.color=}\StringTok{"black"}\NormalTok{)}
\FunctionTok{plot}\NormalTok{(infomap\_communities, video.network, }\AttributeTok{main=}\StringTok{"Infomap Algorithm"}\NormalTok{, }\AttributeTok{vertex.size=}\DecValTok{10}\NormalTok{, }\AttributeTok{edge.arrow.size=}\FloatTok{0.05}\NormalTok{, }\AttributeTok{vertex.label.cex=}\FloatTok{0.7}\NormalTok{, }\AttributeTok{vertex.label.color=}\StringTok{"black"}\NormalTok{)}

\DocumentationTok{\#\# DVD}
\CommentTok{\# Walktrap Algorithm}
\NormalTok{walktrap\_communities\_dvd }\OtherTok{\textless{}{-}} \FunctionTok{cluster\_walktrap}\NormalTok{(dvd.network)}
\FunctionTok{cat}\NormalTok{(}\StringTok{"Walktrap Algorithm:}\SpecialCharTok{\textbackslash{}n}\StringTok{"}\NormalTok{)}
\FunctionTok{print}\NormalTok{(}\FunctionTok{membership}\NormalTok{(walktrap\_communities\_dvd))}
\FunctionTok{cat}\NormalTok{(}\StringTok{"Modularity:"}\NormalTok{, }\FunctionTok{modularity}\NormalTok{(walktrap\_communities\_dvd), }\StringTok{"}\SpecialCharTok{\textbackslash{}n}\StringTok{"}\NormalTok{)}

\CommentTok{\# Infomap Algorithm}
\NormalTok{infomap\_communities\_dvd }\OtherTok{\textless{}{-}} \FunctionTok{cluster\_infomap}\NormalTok{(dvd.network)}
\FunctionTok{cat}\NormalTok{(}\StringTok{"Infomap Algorithm:}\SpecialCharTok{\textbackslash{}n}\StringTok{"}\NormalTok{)}
\FunctionTok{print}\NormalTok{(}\FunctionTok{membership}\NormalTok{(infomap\_communities\_dvd))}
\FunctionTok{cat}\NormalTok{(}\StringTok{"Modularity:"}\NormalTok{, }\FunctionTok{modularity}\NormalTok{(infomap\_communities\_dvd), }\StringTok{"}\SpecialCharTok{\textbackslash{}n}\StringTok{"}\NormalTok{)}

\CommentTok{\# Visualize communities}
\FunctionTok{plot}\NormalTok{(walktrap\_communities\_dvd, dvd.network, }\AttributeTok{main=}\StringTok{"Walktrap Algorithm"}\NormalTok{, }\AttributeTok{vertex.size=}\DecValTok{10}\NormalTok{, }\AttributeTok{edge.arrow.size=}\FloatTok{0.05}\NormalTok{, }\AttributeTok{vertex.label.cex=}\FloatTok{0.7}\NormalTok{, }\AttributeTok{vertex.label.color=}\StringTok{"black"}\NormalTok{)}
\FunctionTok{plot}\NormalTok{(infomap\_communities\_dvd, dvd.network, }\AttributeTok{main=}\StringTok{"Infomap Algorithm"}\NormalTok{, }\AttributeTok{vertex.size=}\DecValTok{10}\NormalTok{, }\AttributeTok{edge.arrow.size=}\FloatTok{0.05}\NormalTok{, }\AttributeTok{vertex.label.cex=}\FloatTok{0.7}\NormalTok{, }\AttributeTok{vertex.label.color=}\StringTok{"black"}\NormalTok{)}

\DocumentationTok{\#\#\#\# Adjacency Matrix}
\CommentTok{\# Create the adjacency matrix}
\NormalTok{adj\_matrix }\OtherTok{\textless{}{-}} \FunctionTok{as\_adjacency\_matrix}\NormalTok{(music.network, }\AttributeTok{sparse =} \ConstantTok{FALSE}\NormalTok{)}

\CommentTok{\# Transpose the adjacency matrix to swap rows and columns}
\NormalTok{adj\_matrix\_transposed }\OtherTok{\textless{}{-}} \FunctionTok{t}\NormalTok{(adj\_matrix)}

\CommentTok{\# Set up the plot with customizations}
\FunctionTok{par}\NormalTok{(}\AttributeTok{mar =} \FunctionTok{c}\NormalTok{(}\DecValTok{5}\NormalTok{, }\DecValTok{5}\NormalTok{, }\DecValTok{4}\NormalTok{, }\DecValTok{2}\NormalTok{) }\SpecialCharTok{+} \FloatTok{0.1}\NormalTok{)  }\CommentTok{\# Increase margins for axis labels}

\CommentTok{\# Visualize the transposed adjacency matrix with gridlines and node labels}
\FunctionTok{image}\NormalTok{(}\DecValTok{1}\SpecialCharTok{:}\FunctionTok{nrow}\NormalTok{(adj\_matrix\_transposed), }\DecValTok{1}\SpecialCharTok{:}\FunctionTok{ncol}\NormalTok{(adj\_matrix\_transposed), adj\_matrix\_transposed, }
      \AttributeTok{main =} \StringTok{"Adjacency Matrix For Music"}\NormalTok{, }
      \AttributeTok{xlab =} \StringTok{"Nodes"}\NormalTok{, }\AttributeTok{ylab =} \StringTok{"Nodes"}\NormalTok{, }
      \AttributeTok{axes =} \ConstantTok{FALSE}\NormalTok{, }\AttributeTok{col =} \FunctionTok{c}\NormalTok{(}\StringTok{"white"}\NormalTok{, }\StringTok{"black"}\NormalTok{))}

\CommentTok{\# Add gridlines}
\FunctionTok{grid}\NormalTok{(}\AttributeTok{nx =} \FunctionTok{nrow}\NormalTok{(adj\_matrix\_transposed), }\AttributeTok{ny =} \FunctionTok{ncol}\NormalTok{(adj\_matrix\_transposed), }\AttributeTok{col =} \StringTok{"gray"}\NormalTok{, }\AttributeTok{lty =} \StringTok{"dotted"}\NormalTok{)}

\CommentTok{\# Add axis labels with the correct node names}
\FunctionTok{axis}\NormalTok{(}\DecValTok{1}\NormalTok{, }\AttributeTok{at =} \DecValTok{1}\SpecialCharTok{:}\FunctionTok{nrow}\NormalTok{(adj\_matrix\_transposed), }\AttributeTok{labels =} \FunctionTok{rownames}\NormalTok{(adj\_matrix\_transposed), }\AttributeTok{las =} \DecValTok{2}\NormalTok{, }\AttributeTok{cex.axis =} \FloatTok{0.7}\NormalTok{, }\AttributeTok{tick =} \ConstantTok{FALSE}\NormalTok{)}
\FunctionTok{axis}\NormalTok{(}\DecValTok{2}\NormalTok{, }\AttributeTok{at =} \DecValTok{1}\SpecialCharTok{:}\FunctionTok{ncol}\NormalTok{(adj\_matrix\_transposed), }\AttributeTok{labels =} \FunctionTok{colnames}\NormalTok{(adj\_matrix\_transposed), }\AttributeTok{las =} \DecValTok{2}\NormalTok{, }\AttributeTok{cex.axis =} \FloatTok{0.7}\NormalTok{, }\AttributeTok{tick =} \ConstantTok{FALSE}\NormalTok{)}

\DocumentationTok{\#\#\#\# Re{-}ordering}
\CommentTok{\# Hierarchical clustering to reorder the adjacency matrix}
\NormalTok{d }\OtherTok{\textless{}{-}} \FunctionTok{dist}\NormalTok{(adj\_matrix) }\CommentTok{\# Distance matrix}
\NormalTok{hc }\OtherTok{\textless{}{-}} \FunctionTok{hclust}\NormalTok{(d)       }\CommentTok{\# Hierarchical clustering}
\NormalTok{order }\OtherTok{\textless{}{-}}\NormalTok{ hc}\SpecialCharTok{$}\NormalTok{order     }\CommentTok{\# Order of the nodes}

\CommentTok{\# Reorder the adjacency matrix}
\NormalTok{adj\_matrix\_reordered }\OtherTok{\textless{}{-}}\NormalTok{ adj\_matrix[order, order]}

\CommentTok{\# Set up the plot with customizations}
\FunctionTok{par}\NormalTok{(}\AttributeTok{mar =} \FunctionTok{c}\NormalTok{(}\DecValTok{5}\NormalTok{, }\DecValTok{5}\NormalTok{, }\DecValTok{4}\NormalTok{, }\DecValTok{2}\NormalTok{) }\SpecialCharTok{+} \FloatTok{0.1}\NormalTok{)  }\CommentTok{\# Increase margins for axis labels}

\CommentTok{\# Plot the reordered adjacency matrix with gridlines and node labels}
\FunctionTok{image}\NormalTok{(}\DecValTok{1}\SpecialCharTok{:}\FunctionTok{nrow}\NormalTok{(adj\_matrix\_reordered), }\DecValTok{1}\SpecialCharTok{:}\FunctionTok{ncol}\NormalTok{(adj\_matrix\_reordered), adj\_matrix\_reordered, }
      \AttributeTok{main =} \StringTok{"Reordered Adjacency Matrix For Music"}\NormalTok{, }
      \AttributeTok{xlab =} \StringTok{"Nodes"}\NormalTok{, }\AttributeTok{ylab =} \StringTok{"Nodes"}\NormalTok{, }
      \AttributeTok{axes =} \ConstantTok{FALSE}\NormalTok{, }\AttributeTok{col =} \FunctionTok{c}\NormalTok{(}\StringTok{"lightpink"}\NormalTok{, }\StringTok{"darkblue"}\NormalTok{))}

\CommentTok{\# Add gridlines}
\FunctionTok{grid}\NormalTok{(}\AttributeTok{nx =} \FunctionTok{nrow}\NormalTok{(adj\_matrix\_reordered), }\AttributeTok{ny =} \FunctionTok{ncol}\NormalTok{(adj\_matrix\_reordered), }\AttributeTok{col =} \StringTok{"gray"}\NormalTok{, }\AttributeTok{lty =} \StringTok{"dotted"}\NormalTok{)}

\CommentTok{\# Add axis labels}
\FunctionTok{axis}\NormalTok{(}\DecValTok{1}\NormalTok{, }\AttributeTok{at =} \DecValTok{1}\SpecialCharTok{:}\FunctionTok{nrow}\NormalTok{(adj\_matrix\_reordered), }\AttributeTok{labels =} \FunctionTok{rownames}\NormalTok{(adj\_matrix\_reordered), }\AttributeTok{las =} \DecValTok{2}\NormalTok{, }\AttributeTok{cex.axis =} \FloatTok{0.7}\NormalTok{, }\AttributeTok{tick =} \ConstantTok{FALSE}\NormalTok{)}
\FunctionTok{axis}\NormalTok{(}\DecValTok{2}\NormalTok{, }\AttributeTok{at =} \DecValTok{1}\SpecialCharTok{:}\FunctionTok{ncol}\NormalTok{(adj\_matrix\_reordered), }\AttributeTok{labels =} \FunctionTok{colnames}\NormalTok{(adj\_matrix\_reordered), }\AttributeTok{las =} \DecValTok{2}\NormalTok{, }\AttributeTok{cex.axis =} \FloatTok{0.7}\NormalTok{, }\AttributeTok{tick =} \ConstantTok{FALSE}\NormalTok{)}

\DocumentationTok{\#\# Optimal}
\NormalTok{modularity\_value }\OtherTok{\textless{}{-}} \FunctionTok{modularity}\NormalTok{(music.cluster)}
\NormalTok{modularity\_value}

\NormalTok{membership }\OtherTok{\textless{}{-}} \FunctionTok{membership}\NormalTok{(music.cluster)}

\CommentTok{\# Reorder the nodes based on community membership}
\NormalTok{ordered\_indices }\OtherTok{\textless{}{-}} \FunctionTok{order}\NormalTok{(membership)}
\NormalTok{ordered\_adjacency\_matrix }\OtherTok{\textless{}{-}} \FunctionTok{as.matrix}\NormalTok{(}\FunctionTok{get.adjacency}\NormalTok{(music.network))[ordered\_indices, ordered\_indices]}

\CommentTok{\# Plot the reordered adjacency matrix}
\FunctionTok{image}\NormalTok{(}\FunctionTok{t}\NormalTok{(ordered\_adjacency\_matrix), }\AttributeTok{main=}\StringTok{"Reordered Adjacency Matrix for Music Network(Optimal)"}\NormalTok{, }\AttributeTok{xlab=}\StringTok{"Nodes"}\NormalTok{, }\AttributeTok{ylab=}\StringTok{"Nodes"}\NormalTok{, }\AttributeTok{axes=}\ConstantTok{FALSE}\NormalTok{, }\AttributeTok{col=}\FunctionTok{c}\NormalTok{(}\StringTok{"white"}\NormalTok{, }\StringTok{"black"}\NormalTok{))}
\FunctionTok{axis}\NormalTok{(}\DecValTok{1}\NormalTok{, }\AttributeTok{at=}\FunctionTok{seq}\NormalTok{(}\DecValTok{0}\NormalTok{, }\DecValTok{1}\NormalTok{, }\AttributeTok{length.out=}\FunctionTok{length}\NormalTok{(ordered\_indices)), }\AttributeTok{labels=}\FunctionTok{V}\NormalTok{(music.network)}\SpecialCharTok{$}\NormalTok{name[ordered\_indices], }\AttributeTok{las=}\DecValTok{2}\NormalTok{, }\AttributeTok{cex.axis=}\FloatTok{0.7}\NormalTok{)}
\FunctionTok{axis}\NormalTok{(}\DecValTok{2}\NormalTok{, }\AttributeTok{at=}\FunctionTok{seq}\NormalTok{(}\DecValTok{0}\NormalTok{, }\DecValTok{1}\NormalTok{, }\AttributeTok{length.out=}\FunctionTok{length}\NormalTok{(ordered\_indices)), }\AttributeTok{labels=}\FunctionTok{V}\NormalTok{(music.network)}\SpecialCharTok{$}\NormalTok{name[ordered\_indices], }\AttributeTok{las=}\DecValTok{2}\NormalTok{, }\AttributeTok{cex.axis=}\FloatTok{0.7}\NormalTok{)}

\DocumentationTok{\#\# Walktrap}
\NormalTok{communities }\OtherTok{\textless{}{-}} \FunctionTok{cluster\_walktrap}\NormalTok{(music.network)}

\CommentTok{\# Get the membership vector}
\NormalTok{membership }\OtherTok{\textless{}{-}} \FunctionTok{membership}\NormalTok{(communities)}

\CommentTok{\# Calculate modularity to quantify the strength of the detected communities}
\NormalTok{modularity\_value }\OtherTok{\textless{}{-}} \FunctionTok{modularity}\NormalTok{(communities)}
\FunctionTok{print}\NormalTok{(}\FunctionTok{paste}\NormalTok{(}\StringTok{"Modularity:"}\NormalTok{, modularity\_value))}

\CommentTok{\# Identify the number of communities}
\NormalTok{num\_communities }\OtherTok{\textless{}{-}} \FunctionTok{length}\NormalTok{(}\FunctionTok{unique}\NormalTok{(membership))}
\FunctionTok{print}\NormalTok{(}\FunctionTok{paste}\NormalTok{(}\StringTok{"Number of Communities:"}\NormalTok{, num\_communities))}

\CommentTok{\# List the members of each community}
\NormalTok{community\_list }\OtherTok{\textless{}{-}} \FunctionTok{split}\NormalTok{(}\FunctionTok{V}\NormalTok{(music.network)}\SpecialCharTok{$}\NormalTok{name, membership)}
\NormalTok{community\_list}

\CommentTok{\# Reorder the nodes based on community membership}
\NormalTok{ordered\_indices }\OtherTok{\textless{}{-}} \FunctionTok{order}\NormalTok{(membership)}
\NormalTok{ordered\_adjacency\_matrix }\OtherTok{\textless{}{-}} \FunctionTok{as.matrix}\NormalTok{(}\FunctionTok{get.adjacency}\NormalTok{(music.network))[ordered\_indices, ordered\_indices]}

\CommentTok{\# Plot the reordered adjacency matrix}
\FunctionTok{image}\NormalTok{(}\FunctionTok{t}\NormalTok{(ordered\_adjacency\_matrix), }\AttributeTok{main=}\StringTok{"Reordered Adjacency Matrix for Music Network(Walktrap)"}\NormalTok{, }\AttributeTok{xlab=}\StringTok{"Nodes"}\NormalTok{, }\AttributeTok{ylab=}\StringTok{"Nodes"}\NormalTok{, }\AttributeTok{axes=}\ConstantTok{FALSE}\NormalTok{, }\AttributeTok{col=}\FunctionTok{c}\NormalTok{(}\StringTok{"white"}\NormalTok{, }\StringTok{"black"}\NormalTok{))}
\FunctionTok{axis}\NormalTok{(}\DecValTok{1}\NormalTok{, }\AttributeTok{at=}\FunctionTok{seq}\NormalTok{(}\DecValTok{0}\NormalTok{, }\DecValTok{1}\NormalTok{, }\AttributeTok{length.out=}\FunctionTok{length}\NormalTok{(ordered\_indices)), }\AttributeTok{labels=}\FunctionTok{V}\NormalTok{(music.network)}\SpecialCharTok{$}\NormalTok{name[ordered\_indices], }\AttributeTok{las=}\DecValTok{2}\NormalTok{, }\AttributeTok{cex.axis=}\FloatTok{0.7}\NormalTok{)}
\FunctionTok{axis}\NormalTok{(}\DecValTok{2}\NormalTok{, }\AttributeTok{at=}\FunctionTok{seq}\NormalTok{(}\DecValTok{0}\NormalTok{, }\DecValTok{1}\NormalTok{, }\AttributeTok{length.out=}\FunctionTok{length}\NormalTok{(ordered\_indices)), }\AttributeTok{labels=}\FunctionTok{V}\NormalTok{(music.network)}\SpecialCharTok{$}\NormalTok{name[ordered\_indices], }\AttributeTok{las=}\DecValTok{2}\NormalTok{, }\AttributeTok{cex.axis=}\FloatTok{0.7}\NormalTok{)}

\DocumentationTok{\#\# Undirected}
\CommentTok{\# change the network to undirected}
\NormalTok{undirected\_music\_network }\OtherTok{\textless{}{-}} \FunctionTok{as.undirected}\NormalTok{(music.network, }\AttributeTok{mode=}\StringTok{"collapse"}\NormalTok{)}

\CommentTok{\# Detect communities using the walktrap algorithm}
\NormalTok{communities }\OtherTok{\textless{}{-}} \FunctionTok{cluster\_fast\_greedy}\NormalTok{(undirected\_music\_network)}

\NormalTok{membership }\OtherTok{\textless{}{-}} \FunctionTok{membership}\NormalTok{(communities)}

\CommentTok{\# Calculate modularity to quantify the strength of the detected communities}
\NormalTok{modularity\_value }\OtherTok{\textless{}{-}} \FunctionTok{modularity}\NormalTok{(communities)}
\FunctionTok{print}\NormalTok{(}\FunctionTok{paste}\NormalTok{(}\StringTok{"Modularity:"}\NormalTok{, modularity\_value))}

\CommentTok{\# Identify the number of communities}
\NormalTok{num\_communities }\OtherTok{\textless{}{-}} \FunctionTok{length}\NormalTok{(}\FunctionTok{unique}\NormalTok{(membership))}
\FunctionTok{print}\NormalTok{(}\FunctionTok{paste}\NormalTok{(}\StringTok{"Number of Communities:"}\NormalTok{, num\_communities))}

\CommentTok{\# List the members of each community}
\NormalTok{community\_list }\OtherTok{\textless{}{-}} \FunctionTok{split}\NormalTok{(}\FunctionTok{V}\NormalTok{(undirected\_music\_network)}\SpecialCharTok{$}\NormalTok{name, membership)}
\NormalTok{community\_list}



\CommentTok{\# Reorder nodes based on community membership}
\NormalTok{ordered\_indices }\OtherTok{\textless{}{-}} \FunctionTok{order}\NormalTok{(membership)}
\NormalTok{ordered\_adjacency\_matrix }\OtherTok{\textless{}{-}} \FunctionTok{t}\NormalTok{(}\FunctionTok{as.matrix}\NormalTok{(}\FunctionTok{get.adjacency}\NormalTok{(undirected\_music\_network))[ordered\_indices, ordered\_indices])}

\CommentTok{\# Plot reordered adjacency matrix with community colors}
\FunctionTok{image}\NormalTok{(}\DecValTok{1}\SpecialCharTok{:}\FunctionTok{nrow}\NormalTok{((ordered\_adjacency\_matrix)), }\DecValTok{1}\SpecialCharTok{:}\FunctionTok{ncol}\NormalTok{(ordered\_adjacency\_matrix), ordered\_adjacency\_matrix, }\AttributeTok{col=}\FunctionTok{c}\NormalTok{(}\StringTok{"white"}\NormalTok{, }\StringTok{"black"}\NormalTok{), }\AttributeTok{main=}\StringTok{"Reordered Adjacency Matrix for Music Network(Fastgreedy)"}\NormalTok{, }\AttributeTok{axes=}\ConstantTok{FALSE}\NormalTok{, )}
\FunctionTok{axis}\NormalTok{(}\DecValTok{1}\NormalTok{, }\AttributeTok{at=}\DecValTok{1}\SpecialCharTok{:}\FunctionTok{nrow}\NormalTok{(ordered\_adjacency\_matrix), }\AttributeTok{labels=}\FunctionTok{V}\NormalTok{(undirected\_music\_network)}\SpecialCharTok{$}\NormalTok{name[ordered\_indices], }\AttributeTok{las=}\DecValTok{2}\NormalTok{, }\AttributeTok{cex.axis=}\FloatTok{0.5}\NormalTok{)}
\FunctionTok{axis}\NormalTok{(}\DecValTok{2}\NormalTok{, }\AttributeTok{at=}\DecValTok{1}\SpecialCharTok{:}\FunctionTok{ncol}\NormalTok{(ordered\_adjacency\_matrix), }\AttributeTok{labels=}\FunctionTok{V}\NormalTok{(undirected\_music\_network)}\SpecialCharTok{$}\NormalTok{name[ordered\_indices], }\AttributeTok{las=}\DecValTok{2}\NormalTok{, }\AttributeTok{cex.axis=}\FloatTok{0.5}\NormalTok{)}


\DocumentationTok{\#\# Book Original Matrix}
\NormalTok{edges\_list\_B }\OtherTok{\textless{}{-}} \FunctionTok{as\_edgelist}\NormalTok{(book.network, }\AttributeTok{names =} \ConstantTok{TRUE}\NormalTok{)}


\NormalTok{igraph\_network\_B }\OtherTok{\textless{}{-}} \FunctionTok{graph.edgelist}\NormalTok{(edges\_list\_B, }\AttributeTok{directed =} \ConstantTok{TRUE}\NormalTok{)}

\NormalTok{network.adjacency\_B }\OtherTok{\textless{}{-}} \FunctionTok{as\_adj}\NormalTok{(igraph\_network\_B)}

\NormalTok{network.adjacency\_B}

\FunctionTok{image}\NormalTok{(}\FunctionTok{Matrix}\NormalTok{(network.adjacency\_B))}

\CommentTok{\# Create the adjacency matrix}
\NormalTok{adj\_matrix }\OtherTok{\textless{}{-}} \FunctionTok{as\_adjacency\_matrix}\NormalTok{(book.network, }\AttributeTok{sparse =} \ConstantTok{FALSE}\NormalTok{)}

\CommentTok{\# Set up the plot with customizations}
\FunctionTok{par}\NormalTok{(}\AttributeTok{mar =} \FunctionTok{c}\NormalTok{(}\DecValTok{5}\NormalTok{, }\DecValTok{5}\NormalTok{, }\DecValTok{4}\NormalTok{, }\DecValTok{2}\NormalTok{) }\SpecialCharTok{+} \FloatTok{0.1}\NormalTok{)  }\CommentTok{\# Increase margins for axis labels}

\CommentTok{\# Visualize the adjacency matrix with gridlines and node labels}
\FunctionTok{image}\NormalTok{(}\DecValTok{1}\SpecialCharTok{:}\FunctionTok{nrow}\NormalTok{(adj\_matrix), }\DecValTok{1}\SpecialCharTok{:}\FunctionTok{ncol}\NormalTok{(adj\_matrix), adj\_matrix, }
      \AttributeTok{main =} \StringTok{"Adjacency Matrix For Books"}\NormalTok{, }
      \AttributeTok{xlab =} \StringTok{"Nodes"}\NormalTok{, }\AttributeTok{ylab =} \StringTok{"Nodes"}\NormalTok{, }
      \AttributeTok{axes =} \ConstantTok{FALSE}\NormalTok{, }\AttributeTok{col =} \FunctionTok{c}\NormalTok{(}\StringTok{"white"}\NormalTok{, }\StringTok{"black"}\NormalTok{))}


\CommentTok{\# Add axis labels}
\FunctionTok{axis}\NormalTok{(}\DecValTok{1}\NormalTok{, }\AttributeTok{at =} \DecValTok{1}\SpecialCharTok{:}\FunctionTok{nrow}\NormalTok{(adj\_matrix), }\AttributeTok{labels =} \FunctionTok{rownames}\NormalTok{(adj\_matrix), }\AttributeTok{las =} \DecValTok{2}\NormalTok{, }\AttributeTok{cex.axis =} \FloatTok{0.7}\NormalTok{, }\AttributeTok{tick =} \ConstantTok{FALSE}\NormalTok{)}
\FunctionTok{axis}\NormalTok{(}\DecValTok{2}\NormalTok{, }\AttributeTok{at =} \DecValTok{1}\SpecialCharTok{:}\FunctionTok{ncol}\NormalTok{(adj\_matrix), }\AttributeTok{labels =} \FunctionTok{colnames}\NormalTok{(adj\_matrix), }\AttributeTok{las =} \DecValTok{2}\NormalTok{, }\AttributeTok{cex.axis =} \FloatTok{0.7}\NormalTok{, }\AttributeTok{tick =} \ConstantTok{FALSE}\NormalTok{)}

\DocumentationTok{\#\#\# Reorder vertices to highlight patterns.}
\CommentTok{\# Hierarchical clustering to reorder the adjacency matrix}
\NormalTok{d }\OtherTok{\textless{}{-}} \FunctionTok{dist}\NormalTok{(adj\_matrix) }\CommentTok{\# Distance matrix}
\NormalTok{hc }\OtherTok{\textless{}{-}} \FunctionTok{hclust}\NormalTok{(d)       }\CommentTok{\# Hierarchical clustering}
\NormalTok{order }\OtherTok{\textless{}{-}}\NormalTok{ hc}\SpecialCharTok{$}\NormalTok{order     }\CommentTok{\# Order of the nodes}

\CommentTok{\# Reorder the adjacency matrix}
\NormalTok{adj\_matrix\_reordered }\OtherTok{\textless{}{-}}\NormalTok{ adj\_matrix[order, order]}

\CommentTok{\# Set up the plot with customizations}
\FunctionTok{par}\NormalTok{(}\AttributeTok{mar =} \FunctionTok{c}\NormalTok{(}\DecValTok{5}\NormalTok{, }\DecValTok{5}\NormalTok{, }\DecValTok{4}\NormalTok{, }\DecValTok{2}\NormalTok{) }\SpecialCharTok{+} \FloatTok{0.1}\NormalTok{)  }\CommentTok{\# Increase margins for axis labels}

\CommentTok{\# Plot the reordered adjacency matrix with gridlines and node labels}
\FunctionTok{image}\NormalTok{(}\DecValTok{1}\SpecialCharTok{:}\FunctionTok{nrow}\NormalTok{(adj\_matrix\_reordered), }\DecValTok{1}\SpecialCharTok{:}\FunctionTok{ncol}\NormalTok{(adj\_matrix\_reordered), adj\_matrix\_reordered, }
      \AttributeTok{main =} \StringTok{"Reordered Adjacency Matrix For Books"}\NormalTok{, }
      \AttributeTok{xlab =} \StringTok{"Nodes"}\NormalTok{, }\AttributeTok{ylab =} \StringTok{"Nodes"}\NormalTok{, }
      \AttributeTok{axes =} \ConstantTok{FALSE}\NormalTok{, }\AttributeTok{col =} \FunctionTok{c}\NormalTok{(}\StringTok{"lightpink"}\NormalTok{, }\StringTok{"darkblue"}\NormalTok{))}

\CommentTok{\# Add gridlines}
\FunctionTok{grid}\NormalTok{(}\AttributeTok{nx =} \FunctionTok{nrow}\NormalTok{(adj\_matrix\_reordered), }\AttributeTok{ny =} \FunctionTok{ncol}\NormalTok{(adj\_matrix\_reordered), }\AttributeTok{col =} \StringTok{"gray"}\NormalTok{, }\AttributeTok{lty =} \StringTok{"dotted"}\NormalTok{)}

\CommentTok{\# Add axis labels}
\FunctionTok{axis}\NormalTok{(}\DecValTok{1}\NormalTok{, }\AttributeTok{at =} \DecValTok{1}\SpecialCharTok{:}\FunctionTok{nrow}\NormalTok{(adj\_matrix\_reordered), }\AttributeTok{labels =} \FunctionTok{rownames}\NormalTok{(adj\_matrix\_reordered), }\AttributeTok{las =} \DecValTok{2}\NormalTok{, }\AttributeTok{cex.axis =} \FloatTok{0.7}\NormalTok{, }\AttributeTok{tick =} \ConstantTok{FALSE}\NormalTok{)}
\FunctionTok{axis}\NormalTok{(}\DecValTok{2}\NormalTok{, }\AttributeTok{at =} \DecValTok{1}\SpecialCharTok{:}\FunctionTok{ncol}\NormalTok{(adj\_matrix\_reordered), }\AttributeTok{labels =} \FunctionTok{colnames}\NormalTok{(adj\_matrix\_reordered), }\AttributeTok{las =} \DecValTok{2}\NormalTok{, }\AttributeTok{cex.axis =} \FloatTok{0.7}\NormalTok{, }\AttributeTok{tick =} \ConstantTok{FALSE}\NormalTok{)}

\CommentTok{\# Detect communities}
\NormalTok{communities }\OtherTok{\textless{}{-}} \FunctionTok{cluster\_walktrap}\NormalTok{(book.network)}

\CommentTok{\# Get the membership vector}
\NormalTok{membership }\OtherTok{\textless{}{-}} \FunctionTok{membership}\NormalTok{(communities)}

\CommentTok{\# Calculate modularity to quantify the strength of the detected communities}
\NormalTok{modularity\_value }\OtherTok{\textless{}{-}} \FunctionTok{modularity}\NormalTok{(communities)}
\FunctionTok{print}\NormalTok{(}\FunctionTok{paste}\NormalTok{(}\StringTok{"Modularity:"}\NormalTok{, modularity\_value))}

\CommentTok{\# Identify the number of communities}
\NormalTok{num\_communities }\OtherTok{\textless{}{-}} \FunctionTok{length}\NormalTok{(}\FunctionTok{unique}\NormalTok{(membership))}
\FunctionTok{print}\NormalTok{(}\FunctionTok{paste}\NormalTok{(}\StringTok{"Number of Communities:"}\NormalTok{, num\_communities))}

\CommentTok{\# List the members of each community}
\NormalTok{community\_list }\OtherTok{\textless{}{-}} \FunctionTok{split}\NormalTok{(}\FunctionTok{V}\NormalTok{(book.network)}\SpecialCharTok{$}\NormalTok{name, membership)}
\NormalTok{community\_list}

\CommentTok{\# Reorder the nodes based on community membership}
\NormalTok{ordered\_indices }\OtherTok{\textless{}{-}} \FunctionTok{order}\NormalTok{(membership)}
\NormalTok{ordered\_adjacency\_matrix }\OtherTok{\textless{}{-}} \FunctionTok{as.matrix}\NormalTok{(}\FunctionTok{get.adjacency}\NormalTok{(book.network))[ordered\_indices, ordered\_indices]}

\CommentTok{\# Plot the reordered adjacency matrix}
\FunctionTok{image}\NormalTok{(ordered\_adjacency\_matrix, }\AttributeTok{main=}\StringTok{"Reordered Adjacency Matrix Based on Community for Book Network"}\NormalTok{, }\AttributeTok{xlab=}\StringTok{"Nodes"}\NormalTok{, }\AttributeTok{ylab=}\StringTok{"Nodes"}\NormalTok{, }\AttributeTok{axes=}\ConstantTok{FALSE}\NormalTok{, }\AttributeTok{col=}\FunctionTok{c}\NormalTok{(}\StringTok{"white"}\NormalTok{, }\StringTok{"black"}\NormalTok{))}
\FunctionTok{axis}\NormalTok{(}\DecValTok{1}\NormalTok{, }\AttributeTok{at=}\FunctionTok{seq}\NormalTok{(}\DecValTok{0}\NormalTok{, }\DecValTok{1}\NormalTok{, }\AttributeTok{length.out=}\FunctionTok{length}\NormalTok{(ordered\_indices)), }\AttributeTok{labels=}\FunctionTok{V}\NormalTok{(book.network)}\SpecialCharTok{$}\NormalTok{name[ordered\_indices], }\AttributeTok{las=}\DecValTok{2}\NormalTok{, }\AttributeTok{cex.axis=}\FloatTok{0.7}\NormalTok{)}

\CommentTok{\# change the network to undirected}
\NormalTok{undirected\_book\_network }\OtherTok{\textless{}{-}} \FunctionTok{as.undirected}\NormalTok{(book.network, }\AttributeTok{mode=}\StringTok{"collapse"}\NormalTok{)}

\CommentTok{\# Detect communities using the fastgreedy algorithm}
\NormalTok{communities }\OtherTok{\textless{}{-}} \FunctionTok{cluster\_fast\_greedy}\NormalTok{(undirected\_book\_network)}

\CommentTok{\# Get the membership vector}
\NormalTok{membership }\OtherTok{\textless{}{-}} \FunctionTok{membership}\NormalTok{(communities)}

\CommentTok{\# Calculate modularity to quantify the strength of the detected communities}
\NormalTok{modularity\_value }\OtherTok{\textless{}{-}} \FunctionTok{modularity}\NormalTok{(communities)}
\FunctionTok{print}\NormalTok{(}\FunctionTok{paste}\NormalTok{(}\StringTok{"Modularity:"}\NormalTok{, modularity\_value))}

\CommentTok{\# Identify the number of communities}
\NormalTok{num\_communities }\OtherTok{\textless{}{-}} \FunctionTok{length}\NormalTok{(}\FunctionTok{unique}\NormalTok{(membership))}
\FunctionTok{print}\NormalTok{(}\FunctionTok{paste}\NormalTok{(}\StringTok{"Number of Communities:"}\NormalTok{, num\_communities))}

\CommentTok{\# List the members of each community}
\NormalTok{community\_list }\OtherTok{\textless{}{-}} \FunctionTok{split}\NormalTok{(}\FunctionTok{V}\NormalTok{(undirected\_book\_network)}\SpecialCharTok{$}\NormalTok{name, membership)}
\NormalTok{community\_list}

\CommentTok{\# Reorder the nodes based on community membership}
\NormalTok{ordered\_indices }\OtherTok{\textless{}{-}} \FunctionTok{order}\NormalTok{(membership)}
\NormalTok{ordered\_adjacency\_matrix }\OtherTok{\textless{}{-}} \FunctionTok{as.matrix}\NormalTok{(}\FunctionTok{get.adjacency}\NormalTok{(undirected\_book\_network))[ordered\_indices, ordered\_indices]}

\CommentTok{\# Plot the reordered adjacency matrix}
\FunctionTok{image}\NormalTok{(ordered\_adjacency\_matrix, }\AttributeTok{main=}\StringTok{"Reordered Adjacency Matrix Based on Community"}\NormalTok{, }\AttributeTok{xlab=}\StringTok{"Nodes"}\NormalTok{, }\AttributeTok{ylab=}\StringTok{"Nodes"}\NormalTok{, }\AttributeTok{axes=}\ConstantTok{FALSE}\NormalTok{, }\AttributeTok{col=}\FunctionTok{c}\NormalTok{(}\StringTok{"white"}\NormalTok{, }\StringTok{"black"}\NormalTok{))}
\FunctionTok{axis}\NormalTok{(}\DecValTok{1}\NormalTok{, }\AttributeTok{at=}\FunctionTok{seq}\NormalTok{(}\DecValTok{0}\NormalTok{, }\DecValTok{1}\NormalTok{, }\AttributeTok{length.out=}\FunctionTok{length}\NormalTok{(ordered\_indices)), }\AttributeTok{labels=}\FunctionTok{V}\NormalTok{(undirected\_book\_network)}\SpecialCharTok{$}\NormalTok{name[ordered\_indices], }\AttributeTok{las=}\DecValTok{2}\NormalTok{, }\AttributeTok{cex.axis=}\FloatTok{0.7}\NormalTok{)}
\FunctionTok{axis}\NormalTok{(}\DecValTok{2}\NormalTok{, }\AttributeTok{at=}\FunctionTok{seq}\NormalTok{(}\DecValTok{0}\NormalTok{, }\DecValTok{1}\NormalTok{, }\AttributeTok{length.out=}\FunctionTok{length}\NormalTok{(ordered\_indices)), }\AttributeTok{labels=}\FunctionTok{V}\NormalTok{(undirected\_book\_network)}\SpecialCharTok{$}\NormalTok{name[ordered\_indices], }\AttributeTok{las=}\DecValTok{2}\NormalTok{, }\AttributeTok{cex.axis=}\FloatTok{0.7}\NormalTok{)}

\DocumentationTok{\#\# Video}
\NormalTok{edges\_list\_V }\OtherTok{\textless{}{-}} \FunctionTok{as\_edgelist}\NormalTok{(video.network, }\AttributeTok{names =} \ConstantTok{TRUE}\NormalTok{)}


\NormalTok{igraph\_network\_V }\OtherTok{\textless{}{-}} \FunctionTok{graph.edgelist}\NormalTok{(edges\_list\_V, }\AttributeTok{directed =} \ConstantTok{TRUE}\NormalTok{)}

\NormalTok{network.adjacency\_V }\OtherTok{\textless{}{-}} \FunctionTok{as\_adj}\NormalTok{(igraph\_network\_V)}

\NormalTok{network.adjacency\_V}

\FunctionTok{image}\NormalTok{(}\FunctionTok{Matrix}\NormalTok{(network.adjacency\_V))}

\CommentTok{\# Create the adjacency matrix}
\NormalTok{adj\_matrix }\OtherTok{\textless{}{-}} \FunctionTok{as\_adjacency\_matrix}\NormalTok{(video.network, }\AttributeTok{sparse =} \ConstantTok{FALSE}\NormalTok{)}

\CommentTok{\# Set up the plot with customizations}
\FunctionTok{par}\NormalTok{(}\AttributeTok{mar =} \FunctionTok{c}\NormalTok{(}\DecValTok{5}\NormalTok{, }\DecValTok{5}\NormalTok{, }\DecValTok{4}\NormalTok{, }\DecValTok{2}\NormalTok{) }\SpecialCharTok{+} \FloatTok{0.1}\NormalTok{)  }\CommentTok{\# Increase margins for axis labels}

\CommentTok{\# Visualize the adjacency matrix with gridlines and node labels}
\FunctionTok{image}\NormalTok{(}\DecValTok{1}\SpecialCharTok{:}\FunctionTok{nrow}\NormalTok{(adj\_matrix), }\DecValTok{1}\SpecialCharTok{:}\FunctionTok{ncol}\NormalTok{(adj\_matrix), adj\_matrix, }
      \AttributeTok{main =} \StringTok{"Adjacency Matrix For Videos"}\NormalTok{, }
      \AttributeTok{xlab =} \StringTok{"Nodes"}\NormalTok{, }\AttributeTok{ylab =} \StringTok{"Nodes"}\NormalTok{, }
      \AttributeTok{axes =} \ConstantTok{FALSE}\NormalTok{, }\AttributeTok{col =} \FunctionTok{c}\NormalTok{(}\StringTok{"white"}\NormalTok{, }\StringTok{"black"}\NormalTok{))}


\CommentTok{\# Add axis labels}
\FunctionTok{axis}\NormalTok{(}\DecValTok{1}\NormalTok{, }\AttributeTok{at =} \DecValTok{1}\SpecialCharTok{:}\FunctionTok{nrow}\NormalTok{(adj\_matrix), }\AttributeTok{labels =} \FunctionTok{rownames}\NormalTok{(adj\_matrix), }\AttributeTok{las =} \DecValTok{2}\NormalTok{, }\AttributeTok{cex.axis =} \FloatTok{0.7}\NormalTok{, }\AttributeTok{tick =} \ConstantTok{FALSE}\NormalTok{)}
\FunctionTok{axis}\NormalTok{(}\DecValTok{2}\NormalTok{, }\AttributeTok{at =} \DecValTok{1}\SpecialCharTok{:}\FunctionTok{ncol}\NormalTok{(adj\_matrix), }\AttributeTok{labels =} \FunctionTok{colnames}\NormalTok{(adj\_matrix), }\AttributeTok{las =} \DecValTok{2}\NormalTok{, }\AttributeTok{cex.axis =} \FloatTok{0.7}\NormalTok{, }\AttributeTok{tick =} \ConstantTok{FALSE}\NormalTok{)}

\CommentTok{\# Hierarchical clustering to reorder the adjacency matrix}
\NormalTok{d }\OtherTok{\textless{}{-}} \FunctionTok{dist}\NormalTok{(adj\_matrix) }\CommentTok{\# Distance matrix}
\NormalTok{hc }\OtherTok{\textless{}{-}} \FunctionTok{hclust}\NormalTok{(d)       }\CommentTok{\# Hierarchical clustering}
\NormalTok{order }\OtherTok{\textless{}{-}}\NormalTok{ hc}\SpecialCharTok{$}\NormalTok{order     }\CommentTok{\# Order of the nodes}

\CommentTok{\# Reorder the adjacency matrix}
\NormalTok{adj\_matrix\_reordered }\OtherTok{\textless{}{-}}\NormalTok{ adj\_matrix[order, order]}

\CommentTok{\# Set up the plot with customizations}
\FunctionTok{par}\NormalTok{(}\AttributeTok{mar =} \FunctionTok{c}\NormalTok{(}\DecValTok{5}\NormalTok{, }\DecValTok{5}\NormalTok{, }\DecValTok{4}\NormalTok{, }\DecValTok{2}\NormalTok{) }\SpecialCharTok{+} \FloatTok{0.1}\NormalTok{)  }\CommentTok{\# Increase margins for axis labels}

\CommentTok{\# Plot the reordered adjacency matrix with gridlines and node labels}
\FunctionTok{image}\NormalTok{(}\DecValTok{1}\SpecialCharTok{:}\FunctionTok{nrow}\NormalTok{(adj\_matrix\_reordered), }\DecValTok{1}\SpecialCharTok{:}\FunctionTok{ncol}\NormalTok{(adj\_matrix\_reordered), adj\_matrix\_reordered, }
      \AttributeTok{main =} \StringTok{"Reordered Adjacency Matrix For Videos"}\NormalTok{, }
      \AttributeTok{xlab =} \StringTok{"Nodes"}\NormalTok{, }\AttributeTok{ylab =} \StringTok{"Nodes"}\NormalTok{, }
      \AttributeTok{axes =} \ConstantTok{FALSE}\NormalTok{, }\AttributeTok{col =} \FunctionTok{c}\NormalTok{(}\StringTok{"lightpink"}\NormalTok{, }\StringTok{"darkblue"}\NormalTok{))}

\CommentTok{\# Add gridlines}
\FunctionTok{grid}\NormalTok{(}\AttributeTok{nx =} \FunctionTok{nrow}\NormalTok{(adj\_matrix\_reordered), }\AttributeTok{ny =} \FunctionTok{ncol}\NormalTok{(adj\_matrix\_reordered), }\AttributeTok{col =} \StringTok{"gray"}\NormalTok{, }\AttributeTok{lty =} \StringTok{"dotted"}\NormalTok{)}

\CommentTok{\# Add axis labels}
\FunctionTok{axis}\NormalTok{(}\DecValTok{1}\NormalTok{, }\AttributeTok{at =} \DecValTok{1}\SpecialCharTok{:}\FunctionTok{nrow}\NormalTok{(adj\_matrix\_reordered), }\AttributeTok{labels =} \FunctionTok{rownames}\NormalTok{(adj\_matrix\_reordered), }\AttributeTok{las =} \DecValTok{2}\NormalTok{, }\AttributeTok{cex.axis =} \FloatTok{0.7}\NormalTok{, }\AttributeTok{tick =} \ConstantTok{FALSE}\NormalTok{)}
\FunctionTok{axis}\NormalTok{(}\DecValTok{2}\NormalTok{, }\AttributeTok{at =} \DecValTok{1}\SpecialCharTok{:}\FunctionTok{ncol}\NormalTok{(adj\_matrix\_reordered), }\AttributeTok{labels =} \FunctionTok{colnames}\NormalTok{(adj\_matrix\_reordered), }\AttributeTok{las =} \DecValTok{2}\NormalTok{, }\AttributeTok{cex.axis =} \FloatTok{0.7}\NormalTok{, }\AttributeTok{tick =} \ConstantTok{FALSE}\NormalTok{)}

\CommentTok{\# Detect communities}
\NormalTok{communities }\OtherTok{\textless{}{-}} \FunctionTok{cluster\_walktrap}\NormalTok{(video.network)}

\CommentTok{\# Get the membership vector}
\NormalTok{membership }\OtherTok{\textless{}{-}} \FunctionTok{membership}\NormalTok{(communities)}

\CommentTok{\# Calculate modularity to quantify the strength of the detected communities}
\NormalTok{modularity\_value }\OtherTok{\textless{}{-}} \FunctionTok{modularity}\NormalTok{(communities)}
\FunctionTok{print}\NormalTok{(}\FunctionTok{paste}\NormalTok{(}\StringTok{"Modularity:"}\NormalTok{, modularity\_value))}

\CommentTok{\# Identify the number of communities}
\NormalTok{num\_communities }\OtherTok{\textless{}{-}} \FunctionTok{length}\NormalTok{(}\FunctionTok{unique}\NormalTok{(membership))}
\FunctionTok{print}\NormalTok{(}\FunctionTok{paste}\NormalTok{(}\StringTok{"Number of Communities:"}\NormalTok{, num\_communities))}

\CommentTok{\# List the members of each community}
\NormalTok{community\_list }\OtherTok{\textless{}{-}} \FunctionTok{split}\NormalTok{(}\FunctionTok{V}\NormalTok{(video.network)}\SpecialCharTok{$}\NormalTok{name, membership)}
\NormalTok{community\_list}

\CommentTok{\# Reorder the nodes based on community membership}
\NormalTok{ordered\_indices }\OtherTok{\textless{}{-}} \FunctionTok{order}\NormalTok{(membership)}
\NormalTok{ordered\_adjacency\_matrix }\OtherTok{\textless{}{-}} \FunctionTok{as.matrix}\NormalTok{(}\FunctionTok{get.adjacency}\NormalTok{(video.network))[ordered\_indices, ordered\_indices]}

\CommentTok{\# Plot the reordered adjacency matrix}
\FunctionTok{image}\NormalTok{(ordered\_adjacency\_matrix, }\AttributeTok{main=}\StringTok{"Reordered Adjacency Matrix Based on Community for Video Network"}\NormalTok{, }\AttributeTok{xlab=}\StringTok{"Nodes"}\NormalTok{, }\AttributeTok{ylab=}\StringTok{"Nodes"}\NormalTok{, }\AttributeTok{axes=}\ConstantTok{FALSE}\NormalTok{, }\AttributeTok{col=}\FunctionTok{c}\NormalTok{(}\StringTok{"white"}\NormalTok{, }\StringTok{"black"}\NormalTok{))}
\FunctionTok{axis}\NormalTok{(}\DecValTok{1}\NormalTok{, }\AttributeTok{at=}\FunctionTok{seq}\NormalTok{(}\DecValTok{0}\NormalTok{, }\DecValTok{1}\NormalTok{, }\AttributeTok{length.out=}\FunctionTok{length}\NormalTok{(ordered\_indices)), }\AttributeTok{labels=}\FunctionTok{V}\NormalTok{(video.network)}\SpecialCharTok{$}\NormalTok{name[ordered\_indices], }\AttributeTok{las=}\DecValTok{2}\NormalTok{, }\AttributeTok{cex.axis=}\FloatTok{0.7}\NormalTok{)}
\FunctionTok{axis}\NormalTok{(}\DecValTok{2}\NormalTok{, }\AttributeTok{at=}\FunctionTok{seq}\NormalTok{(}\DecValTok{0}\NormalTok{, }\DecValTok{1}\NormalTok{, }\AttributeTok{length.out=}\FunctionTok{length}\NormalTok{(ordered\_indices)), }\AttributeTok{labels=}\FunctionTok{V}\NormalTok{(video.network)}\SpecialCharTok{$}\NormalTok{name[ordered\_indices], }\AttributeTok{las=}\DecValTok{2}\NormalTok{, }\AttributeTok{cex.axis=}\FloatTok{0.7}\NormalTok{)}

\CommentTok{\# change the network to undirected}
\NormalTok{undirected\_video\_network }\OtherTok{\textless{}{-}} \FunctionTok{as.undirected}\NormalTok{(video.network, }\AttributeTok{mode=}\StringTok{"collapse"}\NormalTok{)}

\CommentTok{\# Detect communities using the fastgreedy algorithm}
\NormalTok{communities }\OtherTok{\textless{}{-}} \FunctionTok{cluster\_fast\_greedy}\NormalTok{(undirected\_video\_network)}

\CommentTok{\# Get the membership vector}
\NormalTok{membership }\OtherTok{\textless{}{-}} \FunctionTok{membership}\NormalTok{(communities)}

\CommentTok{\# Calculate modularity to quantify the strength of the detected communities}
\NormalTok{modularity\_value }\OtherTok{\textless{}{-}} \FunctionTok{modularity}\NormalTok{(communities)}
\FunctionTok{print}\NormalTok{(}\FunctionTok{paste}\NormalTok{(}\StringTok{"Modularity:"}\NormalTok{, modularity\_value))}

\CommentTok{\# Identify the number of communities}
\NormalTok{num\_communities }\OtherTok{\textless{}{-}} \FunctionTok{length}\NormalTok{(}\FunctionTok{unique}\NormalTok{(membership))}
\FunctionTok{print}\NormalTok{(}\FunctionTok{paste}\NormalTok{(}\StringTok{"Number of Communities:"}\NormalTok{, num\_communities))}

\CommentTok{\# List the members of each community}
\NormalTok{community\_list }\OtherTok{\textless{}{-}} \FunctionTok{split}\NormalTok{(}\FunctionTok{V}\NormalTok{(undirected\_video\_network)}\SpecialCharTok{$}\NormalTok{name, membership)}
\NormalTok{community\_list}

\CommentTok{\# Reorder the nodes based on community membership}
\NormalTok{ordered\_indices }\OtherTok{\textless{}{-}} \FunctionTok{order}\NormalTok{(membership)}
\NormalTok{ordered\_adjacency\_matrix }\OtherTok{\textless{}{-}} \FunctionTok{as.matrix}\NormalTok{(}\FunctionTok{get.adjacency}\NormalTok{(undirected\_video\_network))[ordered\_indices, ordered\_indices]}

\CommentTok{\# Plot the reordered adjacency matrix}
\FunctionTok{image}\NormalTok{(ordered\_adjacency\_matrix, }\AttributeTok{main=}\StringTok{"Reordered Adjacency Matrix Based on Community"}\NormalTok{, }\AttributeTok{xlab=}\StringTok{"Nodes"}\NormalTok{, }\AttributeTok{ylab=}\StringTok{"Nodes"}\NormalTok{, }\AttributeTok{axes=}\ConstantTok{FALSE}\NormalTok{, }\AttributeTok{col=}\FunctionTok{c}\NormalTok{(}\StringTok{"white"}\NormalTok{, }\StringTok{"black"}\NormalTok{))}
\FunctionTok{axis}\NormalTok{(}\DecValTok{1}\NormalTok{, }\AttributeTok{at=}\FunctionTok{seq}\NormalTok{(}\DecValTok{0}\NormalTok{, }\DecValTok{1}\NormalTok{, }\AttributeTok{length.out=}\FunctionTok{length}\NormalTok{(ordered\_indices)), }\AttributeTok{labels=}\FunctionTok{V}\NormalTok{(undirected\_video\_network)}\SpecialCharTok{$}\NormalTok{name[ordered\_indices], }\AttributeTok{las=}\DecValTok{2}\NormalTok{, }\AttributeTok{cex.axis=}\FloatTok{0.7}\NormalTok{)}
\FunctionTok{axis}\NormalTok{(}\DecValTok{2}\NormalTok{, }\AttributeTok{at=}\FunctionTok{seq}\NormalTok{(}\DecValTok{0}\NormalTok{, }\DecValTok{1}\NormalTok{, }\AttributeTok{length.out=}\FunctionTok{length}\NormalTok{(ordered\_indices)), }\AttributeTok{labels=}\FunctionTok{V}\NormalTok{(undirected\_video\_network)}\SpecialCharTok{$}\NormalTok{name[ordered\_indices], }\AttributeTok{las=}\DecValTok{2}\NormalTok{, }\AttributeTok{cex.axis=}\FloatTok{0.7}\NormalTok{)}

\NormalTok{edges\_list\_D }\OtherTok{\textless{}{-}} \FunctionTok{as\_edgelist}\NormalTok{(dvd.network, }\AttributeTok{names =} \ConstantTok{TRUE}\NormalTok{)}


\NormalTok{igraph\_network\_D }\OtherTok{\textless{}{-}} \FunctionTok{graph.edgelist}\NormalTok{(edges\_list\_D, }\AttributeTok{directed =} \ConstantTok{TRUE}\NormalTok{)}

\NormalTok{network.adjacency\_D }\OtherTok{\textless{}{-}} \FunctionTok{as\_adj}\NormalTok{(igraph\_network\_D)}

\NormalTok{network.adjacency\_D}

\FunctionTok{image}\NormalTok{(}\FunctionTok{Matrix}\NormalTok{(network.adjacency\_D))}

\CommentTok{\# Create the adjacency matrix}
\NormalTok{adj\_matrix }\OtherTok{\textless{}{-}} \FunctionTok{as\_adjacency\_matrix}\NormalTok{(dvd.network, }\AttributeTok{sparse =} \ConstantTok{FALSE}\NormalTok{)}

\CommentTok{\# Set up the plot with customizations}
\FunctionTok{par}\NormalTok{(}\AttributeTok{mar =} \FunctionTok{c}\NormalTok{(}\DecValTok{5}\NormalTok{, }\DecValTok{5}\NormalTok{, }\DecValTok{4}\NormalTok{, }\DecValTok{2}\NormalTok{) }\SpecialCharTok{+} \FloatTok{0.1}\NormalTok{)  }\CommentTok{\# Increase margins for axis labels}

\CommentTok{\# Visualize the adjacency matrix with gridlines and node labels}
\FunctionTok{image}\NormalTok{(}\DecValTok{1}\SpecialCharTok{:}\FunctionTok{nrow}\NormalTok{(adj\_matrix), }\DecValTok{1}\SpecialCharTok{:}\FunctionTok{ncol}\NormalTok{(adj\_matrix), adj\_matrix, }
      \AttributeTok{main =} \StringTok{"Adjacency Matrix For DVD"}\NormalTok{, }
      \AttributeTok{xlab =} \StringTok{"Nodes"}\NormalTok{, }\AttributeTok{ylab =} \StringTok{"Nodes"}\NormalTok{, }
      \AttributeTok{axes =} \ConstantTok{FALSE}\NormalTok{, }\AttributeTok{col =} \FunctionTok{c}\NormalTok{(}\StringTok{"white"}\NormalTok{, }\StringTok{"black"}\NormalTok{))}



\CommentTok{\# Add axis labels}
\FunctionTok{axis}\NormalTok{(}\DecValTok{1}\NormalTok{, }\AttributeTok{at =} \DecValTok{1}\SpecialCharTok{:}\FunctionTok{nrow}\NormalTok{(adj\_matrix), }\AttributeTok{labels =} \FunctionTok{rownames}\NormalTok{(adj\_matrix), }\AttributeTok{las =} \DecValTok{2}\NormalTok{, }\AttributeTok{cex.axis =} \FloatTok{0.7}\NormalTok{, }\AttributeTok{tick =} \ConstantTok{FALSE}\NormalTok{)}
\FunctionTok{axis}\NormalTok{(}\DecValTok{2}\NormalTok{, }\AttributeTok{at =} \DecValTok{1}\SpecialCharTok{:}\FunctionTok{ncol}\NormalTok{(adj\_matrix), }\AttributeTok{labels =} \FunctionTok{colnames}\NormalTok{(adj\_matrix), }\AttributeTok{las =} \DecValTok{2}\NormalTok{, }\AttributeTok{cex.axis =} \FloatTok{0.7}\NormalTok{, }\AttributeTok{tick =} \ConstantTok{FALSE}\NormalTok{)}

\DocumentationTok{\#\#\# Reorder vertices to highlight patterns.}
\CommentTok{\# Hierarchical clustering to reorder the adjacency matrix}
\NormalTok{d }\OtherTok{\textless{}{-}} \FunctionTok{dist}\NormalTok{(adj\_matrix) }\CommentTok{\# Distance matrix}
\NormalTok{hc }\OtherTok{\textless{}{-}} \FunctionTok{hclust}\NormalTok{(d)       }\CommentTok{\# Hierarchical clustering}
\NormalTok{order }\OtherTok{\textless{}{-}}\NormalTok{ hc}\SpecialCharTok{$}\NormalTok{order     }\CommentTok{\# Order of the nodes}

\CommentTok{\# Reorder the adjacency matrix}
\NormalTok{adj\_matrix\_reordered }\OtherTok{\textless{}{-}}\NormalTok{ adj\_matrix[order, order]}

\CommentTok{\# Set up the plot with customizations}
\FunctionTok{par}\NormalTok{(}\AttributeTok{mar =} \FunctionTok{c}\NormalTok{(}\DecValTok{5}\NormalTok{, }\DecValTok{5}\NormalTok{, }\DecValTok{4}\NormalTok{, }\DecValTok{2}\NormalTok{) }\SpecialCharTok{+} \FloatTok{0.1}\NormalTok{)  }\CommentTok{\# Increase margins for axis labels}

\CommentTok{\# Plot the reordered adjacency matrix with gridlines and node labels}
\FunctionTok{image}\NormalTok{(}\DecValTok{1}\SpecialCharTok{:}\FunctionTok{nrow}\NormalTok{(adj\_matrix\_reordered), }\DecValTok{1}\SpecialCharTok{:}\FunctionTok{ncol}\NormalTok{(adj\_matrix\_reordered), adj\_matrix\_reordered, }
      \AttributeTok{main =} \StringTok{"Reordered Adjacency Matrix For DVD"}\NormalTok{, }
      \AttributeTok{xlab =} \StringTok{"Nodes"}\NormalTok{, }\AttributeTok{ylab =} \StringTok{"Nodes"}\NormalTok{, }
      \AttributeTok{axes =} \ConstantTok{FALSE}\NormalTok{, }\AttributeTok{col =} \FunctionTok{c}\NormalTok{(}\StringTok{"lightpink"}\NormalTok{, }\StringTok{"darkblue"}\NormalTok{))}

\CommentTok{\# Add gridlines}
\FunctionTok{grid}\NormalTok{(}\AttributeTok{nx =} \FunctionTok{nrow}\NormalTok{(adj\_matrix\_reordered), }\AttributeTok{ny =} \FunctionTok{ncol}\NormalTok{(adj\_matrix\_reordered), }\AttributeTok{col =} \StringTok{"gray"}\NormalTok{, }\AttributeTok{lty =} \StringTok{"dotted"}\NormalTok{)}

\CommentTok{\# Add axis labels}
\FunctionTok{axis}\NormalTok{(}\DecValTok{1}\NormalTok{, }\AttributeTok{at =} \DecValTok{1}\SpecialCharTok{:}\FunctionTok{nrow}\NormalTok{(adj\_matrix\_reordered), }\AttributeTok{labels =} \FunctionTok{rownames}\NormalTok{(adj\_matrix\_reordered), }\AttributeTok{las =} \DecValTok{2}\NormalTok{, }\AttributeTok{cex.axis =} \FloatTok{0.7}\NormalTok{, }\AttributeTok{tick =} \ConstantTok{FALSE}\NormalTok{)}
\FunctionTok{axis}\NormalTok{(}\DecValTok{2}\NormalTok{, }\AttributeTok{at =} \DecValTok{1}\SpecialCharTok{:}\FunctionTok{ncol}\NormalTok{(adj\_matrix\_reordered), }\AttributeTok{labels =} \FunctionTok{colnames}\NormalTok{(adj\_matrix\_reordered), }\AttributeTok{las =} \DecValTok{2}\NormalTok{, }\AttributeTok{cex.axis =} \FloatTok{0.7}\NormalTok{, }\AttributeTok{tick =} \ConstantTok{FALSE}\NormalTok{)}

\CommentTok{\# Detect communities}
\NormalTok{communities }\OtherTok{\textless{}{-}} \FunctionTok{cluster\_walktrap}\NormalTok{(dvd.network)}

\CommentTok{\# Get the membership vector}
\NormalTok{membership }\OtherTok{\textless{}{-}} \FunctionTok{membership}\NormalTok{(communities)}

\CommentTok{\# Calculate modularity to quantify the strength of the detected communities}
\NormalTok{modularity\_value }\OtherTok{\textless{}{-}} \FunctionTok{modularity}\NormalTok{(communities)}
\FunctionTok{print}\NormalTok{(}\FunctionTok{paste}\NormalTok{(}\StringTok{"Modularity:"}\NormalTok{, modularity\_value))}

\CommentTok{\# Identify the number of communities}
\NormalTok{num\_communities }\OtherTok{\textless{}{-}} \FunctionTok{length}\NormalTok{(}\FunctionTok{unique}\NormalTok{(membership))}
\FunctionTok{print}\NormalTok{(}\FunctionTok{paste}\NormalTok{(}\StringTok{"Number of Communities:"}\NormalTok{, num\_communities))}

\CommentTok{\# List the members of each community}
\NormalTok{community\_list }\OtherTok{\textless{}{-}} \FunctionTok{split}\NormalTok{(}\FunctionTok{V}\NormalTok{(dvd.network)}\SpecialCharTok{$}\NormalTok{name, membership)}
\NormalTok{community\_list}

\CommentTok{\# Reorder the nodes based on community membership}
\NormalTok{ordered\_indices }\OtherTok{\textless{}{-}} \FunctionTok{order}\NormalTok{(membership)}
\NormalTok{ordered\_adjacency\_matrix }\OtherTok{\textless{}{-}} \FunctionTok{as.matrix}\NormalTok{(}\FunctionTok{get.adjacency}\NormalTok{(dvd.network))[ordered\_indices, ordered\_indices]}

\CommentTok{\# Plot the reordered adjacency matrix}
\FunctionTok{image}\NormalTok{(ordered\_adjacency\_matrix, }\AttributeTok{main=}\StringTok{"Reordered Adjacency Matrix Based on Community for DVD Network"}\NormalTok{, }\AttributeTok{xlab=}\StringTok{"Nodes"}\NormalTok{, }\AttributeTok{ylab=}\StringTok{"Nodes"}\NormalTok{, }\AttributeTok{axes=}\ConstantTok{FALSE}\NormalTok{, }\AttributeTok{col=}\FunctionTok{c}\NormalTok{(}\StringTok{"white"}\NormalTok{, }\StringTok{"black"}\NormalTok{))}
\FunctionTok{axis}\NormalTok{(}\DecValTok{1}\NormalTok{, }\AttributeTok{at=}\FunctionTok{seq}\NormalTok{(}\DecValTok{0}\NormalTok{, }\DecValTok{1}\NormalTok{, }\AttributeTok{length.out=}\FunctionTok{length}\NormalTok{(ordered\_indices)), }\AttributeTok{labels=}\FunctionTok{V}\NormalTok{(dvd.network)}\SpecialCharTok{$}\NormalTok{name[ordered\_indices], }\AttributeTok{las=}\DecValTok{2}\NormalTok{, }\AttributeTok{cex.axis=}\FloatTok{0.7}\NormalTok{)}
\FunctionTok{axis}\NormalTok{(}\DecValTok{2}\NormalTok{, }\AttributeTok{at=}\FunctionTok{seq}\NormalTok{(}\DecValTok{0}\NormalTok{, }\DecValTok{1}\NormalTok{, }\AttributeTok{length.out=}\FunctionTok{length}\NormalTok{(ordered\_indices)), }\AttributeTok{labels=}\FunctionTok{V}\NormalTok{(dvd.network)}\SpecialCharTok{$}\NormalTok{name[ordered\_indices], }\AttributeTok{las=}\DecValTok{2}\NormalTok{, }\AttributeTok{cex.axis=}\FloatTok{0.7}\NormalTok{)}

\CommentTok{\# change the network to undirected}
\NormalTok{undirected\_dvd\_network }\OtherTok{\textless{}{-}} \FunctionTok{as.undirected}\NormalTok{(dvd.network, }\AttributeTok{mode=}\StringTok{"collapse"}\NormalTok{)}

\CommentTok{\# Detect communities using the fastgreedy algorithm}
\NormalTok{communities }\OtherTok{\textless{}{-}} \FunctionTok{cluster\_fast\_greedy}\NormalTok{(undirected\_dvd\_network)}

\CommentTok{\# Get the membership vector}
\NormalTok{membership }\OtherTok{\textless{}{-}} \FunctionTok{membership}\NormalTok{(communities)}

\CommentTok{\# Calculate modularity to quantify the strength of the detected communities}
\NormalTok{modularity\_value }\OtherTok{\textless{}{-}} \FunctionTok{modularity}\NormalTok{(communities)}
\FunctionTok{print}\NormalTok{(}\FunctionTok{paste}\NormalTok{(}\StringTok{"Modularity:"}\NormalTok{, modularity\_value))}

\CommentTok{\# Identify the number of communities}
\NormalTok{num\_communities }\OtherTok{\textless{}{-}} \FunctionTok{length}\NormalTok{(}\FunctionTok{unique}\NormalTok{(membership))}
\FunctionTok{print}\NormalTok{(}\FunctionTok{paste}\NormalTok{(}\StringTok{"Number of Communities:"}\NormalTok{, num\_communities))}

\CommentTok{\# List the members of each community}
\NormalTok{community\_list }\OtherTok{\textless{}{-}} \FunctionTok{split}\NormalTok{(}\FunctionTok{V}\NormalTok{(undirected\_dvd\_network)}\SpecialCharTok{$}\NormalTok{name, membership)}
\NormalTok{community\_list}

\CommentTok{\# Reorder the nodes based on community membership}
\NormalTok{ordered\_indices }\OtherTok{\textless{}{-}} \FunctionTok{order}\NormalTok{(membership)}
\NormalTok{ordered\_adjacency\_matrix }\OtherTok{\textless{}{-}} \FunctionTok{as.matrix}\NormalTok{(}\FunctionTok{get.adjacency}\NormalTok{(undirected\_dvd\_network))[ordered\_indices, ordered\_indices]}

\CommentTok{\# Plot the reordered adjacency matrix}
\FunctionTok{image}\NormalTok{(ordered\_adjacency\_matrix, }\AttributeTok{main=}\StringTok{"Reordered Adjacency Matrix Based on Community"}\NormalTok{, }\AttributeTok{xlab=}\StringTok{"Nodes"}\NormalTok{, }\AttributeTok{ylab=}\StringTok{"Nodes"}\NormalTok{, }\AttributeTok{axes=}\ConstantTok{FALSE}\NormalTok{, }\AttributeTok{col=}\FunctionTok{c}\NormalTok{(}\StringTok{"white"}\NormalTok{, }\StringTok{"black"}\NormalTok{))}
\FunctionTok{axis}\NormalTok{(}\DecValTok{1}\NormalTok{, }\AttributeTok{at=}\FunctionTok{seq}\NormalTok{(}\DecValTok{0}\NormalTok{, }\DecValTok{1}\NormalTok{, }\AttributeTok{length.out=}\FunctionTok{length}\NormalTok{(ordered\_indices)), }\AttributeTok{labels=}\FunctionTok{V}\NormalTok{(undirected\_dvd\_network)}\SpecialCharTok{$}\NormalTok{name[ordered\_indices], }\AttributeTok{las=}\DecValTok{2}\NormalTok{, }\AttributeTok{cex.axis=}\FloatTok{0.7}\NormalTok{)}
\FunctionTok{axis}\NormalTok{(}\DecValTok{2}\NormalTok{, }\AttributeTok{at=}\FunctionTok{seq}\NormalTok{(}\DecValTok{0}\NormalTok{, }\DecValTok{1}\NormalTok{, }\AttributeTok{length.out=}\FunctionTok{length}\NormalTok{(ordered\_indices)), }\AttributeTok{labels=}\FunctionTok{V}\NormalTok{(undirected\_dvd\_network)}\SpecialCharTok{$}\NormalTok{name[ordered\_indices], }\AttributeTok{las=}\DecValTok{2}\NormalTok{, }\AttributeTok{cex.axis=}\FloatTok{0.7}\NormalTok{)}

\NormalTok{edges\_list\_F }\OtherTok{\textless{}{-}} \FunctionTok{as\_edgelist}\NormalTok{(a, }\AttributeTok{names =} \ConstantTok{TRUE}\NormalTok{)}


\NormalTok{igraph\_network\_F }\OtherTok{\textless{}{-}} \FunctionTok{graph.edgelist}\NormalTok{(edges\_list\_F, }\AttributeTok{directed =} \ConstantTok{TRUE}\NormalTok{)}

\NormalTok{network.adjacency\_F }\OtherTok{\textless{}{-}} \FunctionTok{as\_adj}\NormalTok{(igraph\_network\_F)}

\NormalTok{network.adjacency\_F}

\FunctionTok{image}\NormalTok{(}\FunctionTok{Matrix}\NormalTok{(network.adjacency\_F))}

\NormalTok{com }\OtherTok{\textless{}{-}} \FunctionTok{cluster\_walktrap}\NormalTok{(a)}
\FunctionTok{plot}\NormalTok{(com, a, }\AttributeTok{main=}\StringTok{"Walktrap Algorithm"}\NormalTok{, }\AttributeTok{vertex.size=}\DecValTok{10}\NormalTok{, }\AttributeTok{edge.arrow.size=}\FloatTok{0.05}\NormalTok{, }\AttributeTok{vertex.label.cex=}\FloatTok{0.7}\NormalTok{, }\AttributeTok{vertex.label.color=}\StringTok{"black"}\NormalTok{)}

\CommentTok{\# Get the membership vector}
\NormalTok{mem }\OtherTok{\textless{}{-}} \FunctionTok{membership}\NormalTok{(com)}

\CommentTok{\# List the members of each community}
\NormalTok{community\_list }\OtherTok{\textless{}{-}} \FunctionTok{split}\NormalTok{(}\FunctionTok{V}\NormalTok{(a)}\SpecialCharTok{$}\NormalTok{name, mem)}
\NormalTok{community\_list}

\CommentTok{\# Reorder the nodes based on community membership}
\NormalTok{ordered\_indices }\OtherTok{\textless{}{-}} \FunctionTok{order}\NormalTok{(mem)}
\NormalTok{ordered\_adjacency\_matrix }\OtherTok{\textless{}{-}} \FunctionTok{as.matrix}\NormalTok{(}\FunctionTok{get.adjacency}\NormalTok{(a))[ordered\_indices, ordered\_indices]}

\CommentTok{\# Plot the reordered adjacency matrix}
\FunctionTok{image}\NormalTok{(ordered\_adjacency\_matrix, }\AttributeTok{main=}\StringTok{"Reordered Adjacency Matrix"}\NormalTok{, }\AttributeTok{xlab=}\StringTok{"Nodes"}\NormalTok{, }\AttributeTok{ylab=}\StringTok{"Nodes"}\NormalTok{, }\AttributeTok{axes=}\ConstantTok{FALSE}\NormalTok{, }\AttributeTok{col=}\FunctionTok{c}\NormalTok{(}\StringTok{"white"}\NormalTok{, }\StringTok{"black"}\NormalTok{))}
\FunctionTok{axis}\NormalTok{(}\DecValTok{1}\NormalTok{, }\AttributeTok{at=}\FunctionTok{seq}\NormalTok{(}\DecValTok{0}\NormalTok{, }\DecValTok{1}\NormalTok{, }\AttributeTok{length.out=}\FunctionTok{length}\NormalTok{(ordered\_indices)), }\AttributeTok{labels=}\FunctionTok{V}\NormalTok{(a)}\SpecialCharTok{$}\NormalTok{name[ordered\_indices], }\AttributeTok{las=}\DecValTok{2}\NormalTok{, }\AttributeTok{cex.axis=}\FloatTok{0.7}\NormalTok{)}
\FunctionTok{axis}\NormalTok{(}\DecValTok{2}\NormalTok{, }\AttributeTok{at=}\FunctionTok{seq}\NormalTok{(}\DecValTok{0}\NormalTok{, }\DecValTok{1}\NormalTok{, }\AttributeTok{length.out=}\FunctionTok{length}\NormalTok{(ordered\_indices)), }\AttributeTok{labels=}\FunctionTok{V}\NormalTok{(a)}\SpecialCharTok{$}\NormalTok{name[ordered\_indices], }\AttributeTok{las=}\DecValTok{2}\NormalTok{, }\AttributeTok{cex.axis=}\FloatTok{0.7}\NormalTok{)}
\end{Highlighting}
\end{Shaded}


\end{document}
